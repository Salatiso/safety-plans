\documentclass[11pt]{article}
\usepackage[utf8]{inputenc}
\usepackage[a4paper, margin=2.5cm]{geometry}
\usepackage{fancyhdr}
\usepackage{draftwatermark}
\usepackage{tocbibind}
\usepackage{tabularx}
\usepackage{booktabs}
\usepackage{graphicx}
\usepackage{hyperref}
\usepackage{noto}
\usepackage{ifthen}

% Branding variables
\newcommand{\brandingLevel}{light}
\newcommand{\userLogo}{}
\newcommand{\userCompany}{{{COMPANY_NAME}}}
\newcommand{\safetyHelpLogo}{/safety-plans/assets/images/logo.png}
\SetWatermarkText{SafetyHelp}
\SetWatermarkScale{2}
\SetWatermarkColor[gray]{0.9}

% Dynamic fields
\newcommand{\docTitle}{Emergency Management Standard}
\newcommand{\refNumber}{EMS-001}
\newcommand{\issueDate}{{{DATE}}}
\newcommand{\projectName}{}
\newcommand{\clientName}{{{COMPANY_NAME}}}
\newcommand{\clientContact}{{{COMPANY_CONTACT}}, {{COMPANY_EMAIL}}}
\newcommand{\compilerName}{{{COMPILED_BY_NAME}}}
\newcommand{\compilerRole}{{{COMPILED_BY_ROLE}}}
\newcommand{\compilerSACPCMP}{}
\newcommand{\compilerEmail}{{{COMPILED_BY_EMAIL}}}
\newcommand{\compilerPhone}{{{COMPILED_BY_CONTACT}}}
\newcommand{\compilerCompany}{}
\newcommand{\revision}{{{REVISION}}}
\newcommand{\reviewDate}{{{REVIEW_DATE}}}

% Header and Footer
\pagestyle{fancy}
\fancyhf{}
\ifthenelse{\equal{\brandingLevel}{heavy}}{
  \fancyhead[L]{\small \refNumber}
  \fancyhead[C]{\ifthen\isempty{\userLogo}{\textbf{\userCompany}}{\includegraphics[height=1.5cm]{\userLogo}} \quad \includegraphics[height=1cm]{\safetyHelpLogo}}
  \fancyhead[R]{\small \compilerName}
  \SetWatermarkLightness{0.7}
}{
  \fancyhead[L]{\small \refNumber}
  \fancyhead[C]{\ifthen\isempty{\userLogo}{\textbf{\userCompany}}{\includegraphics[height=1.5cm]{\userLogo}}}
  \fancyhead[R]{\small \compilerName}
}
\ifthenelse{\equal{\brandingLevel}{none}}{
  \SetWatermarkText{}
}{}
\fancyfoot[L]{\small \textit{Powered by SafetyHelp | Safety Personalized, Compliance Simplified}}
\fancyfoot[C]{\small \textit{Visit safetyfirst.help | Email: salatiso@safetyfirst.help}}
\fancyfoot[R]{\small Page \thepage\ of \pageref{LastPage}}
\renewcommand{\headrulewidth}{0.4pt}
\renewcommand{\footrulewidth}{0pt}
\fancyfoot[C]{\parbox[b]{\textwidth}{\footnotesize \textit{This document is generated by SafetyHelp for OHS compliance. Verify applicability with local regulations.}}}

\begin{document}

% Cover Page
\begin{titlepage}
  \centering
  \ifthenelse{\equal{\brandingLevel}{heavy} \OR \equal{\brandingLevel}{light}}{
    \includegraphics[height=1.5cm,keepaspectratio]{\safetyHelpLogo}\hfill
  }{}
  \vspace{2cm}
  \ifthen\isempty{\userLogo}{\textbf{\Large \userCompany}\vspace{1cm}}{\includegraphics[height=1.5cm]{\userLogo}\vspace{1cm}}
  {\Huge \textbf{\docTitle}}\vspace{1cm}
  {\Large Reference: \refNumber}\vspace{0.5cm}
  {\large Issue Date: \issueDate}\vspace{0.5cm}
  {\large Review Date: \reviewDate}\vspace{0.5cm}
  {\large Client: \clientName}\vspace{0.5cm}
  {\large \clientContact}\vspace{0.5cm}
  {\large Compiled By: \compilerName, \compilerRole}\vspace{0.5cm}
  {\large \compilerEmail, \compilerPhone}
  \vspace{2cm}
  \rule{\textwidth}{0.5pt}
\end{titlepage}

% Table of Contents
\tableofcontents
\newpage

% Introduction
\section{Introduction}
This Emergency Management Standard provides the framework and minimum requirements for establishing, implementing, maintaining, and continually improving emergency preparedness and response capabilities across \clientName. It supports the \clientName\ Emergency Management Policy and aims to ensure a consistent and effective approach to managing emergencies.

\subsection{Purpose}
The purpose of this standard is to:
\begin{itemize}
    \item Define the core elements of \clientName's Emergency Management System (EMS).
    \item Outline the responsibilities for implementing and maintaining the EMS.
    \item Ensure compliance with legal, regulatory, and other requirements related to emergency management.
    \item Provide a basis for the development of site-specific Emergency Response Plans (ERPs).
\end{itemize}

\subsection{Scope}
This standard applies to all \clientName\ facilities, operations, employees, contractors, and visitors as defined in the Emergency Management Policy.

% Compiled By Details
\section{Compiled By Details}
\begin{tabularx}{\textwidth}{lX}
  \toprule
  \textbf{Field} & \textbf{Details} \\
  \midrule
  Name & \compilerName \\
  Role & \compilerRole \\
  SACPCMP Number & \compilerSACPCMP \\
  Email & \compilerEmail \\
  Phone & \compilerPhone \\
  Date & \issueDate \\
  Company & \compilerCompany \\
  \bottomrule
\end{tabularx}

% Revision History
\section{Revision History}
\begin{tabularx}{\textwidth}{lXll}
  \toprule
  \textbf{Rev} & \textbf{Date} & \textbf{Changes} & \textbf{Revised By} \\
  \midrule
  \revision & \issueDate & Initial version & \compilerName \\
  \bottomrule
\end{tabularx}

% Content-Specific Sections
\section{Definitions}
Refer to the \clientName\ Emergency Management Policy for key definitions. Additional definitions specific to this standard include:
\begin{itemize}
    \item \textbf{Emergency Control Centre (ECC):} A designated location from which emergency response operations are coordinated.
    \item \textbf{Drill:} A supervised exercise involving a simulated emergency scenario to evaluate the effectiveness of the ERP.
    \item \textbf{Hot Work:} Any work that involves open flames or produces heat and sparks, such as welding, cutting, grinding.
    \item \textbf{Hazard Identification and Risk Assessment (HIRA):} A systematic process to identify potential emergency scenarios and evaluate their likelihood and potential consequences.
\end{itemize}

\section{Legal and Standard References}
This standard is based on and aims to comply with:
\begin{itemize}
    \item \textbf{South African Legislation:}
    \begin{itemize}
        \item Constitution of the Republic of South Africa, 1996 (Section 24).
        \item Occupational Health and Safety Act 85 of 1993 and its applicable regulations.
    \end{itemize}
    \item \textbf{South African National Standards (SANS):}
    \begin{itemize}
        \item SANS 45001: Occupational health and safety management systems.
        \item SANS 10400-T: Fire Protection.
    \end{itemize}
    \item \textbf{Municipal By-laws:} Applicable local emergency services and fire safety by-laws.
\end{itemize}

\section{Key Elements of the Emergency Management System}

\subsection{Leadership and Commitment}
Top management shall demonstrate leadership and commitment to emergency management as outlined in the Emergency Management Policy.

\subsection{Planning}

\subsubsection{Hazard Identification and Risk Assessment (HIRA) for Emergencies}
A systematic HIRA process shall be established to identify potential emergency scenarios (e.g., fire, chemical spill, natural disaster). The HIRA shall consider:
\begin{itemize}
    \item Nature of work activities, processes, and materials used.
    \item Building design, layout, construction, and occupancy.
    \item Location and surrounding environment.
    \item Historical incident data.
    \item Vulnerable persons.
    \item Applicable legal requirements.
\end{itemize}

\subsubsection{Emergency Response Plan (ERP) Development}
A documented ERP shall be developed for each \clientName\ site/facility, based on the HIRA findings, and aligned with this Standard.

\subsection{Support}

\subsubsection{Resources}
\clientName\ shall provide necessary resources for the EMS, including competent personnel, emergency equipment, and financial resources.

\subsubsection{Roles, Responsibilities, and Authorities}
Roles shall be defined, including the appointment of an Emergency Response Team (ERT) with roles such as Emergency Coordinator, Fire Wardens, and First Aiders.

\subsubsection{Competence, Training, and Awareness}
All employees shall receive general emergency awareness training, and ERT members shall receive specialized training appropriate to their roles.

\subsubsection{Communication}
Procedures for internal and external communication during emergencies shall be established, including alarm systems and contact lists.

\subsection{Operation}

\subsubsection{Operational Planning and Control for Emergencies}
Procedures shall be implemented to control situations that could lead to emergencies, and emergency equipment shall be maintained and accessible.

\subsubsection{Emergency Drills and Exercises}
Regular drills shall be conducted to test the effectiveness of the ERP, with documented evaluations and corrective actions.

\subsection{Performance Evaluation}
The EMS shall be monitored through drills, audits, and management reviews to ensure continual improvement.

\subsection{Improvement}
Incidents and nonconformities shall be investigated, with corrective actions implemented to prevent recurrence.

\section{Document Control and Record Keeping}
All EMS documentation shall be controlled in accordance with \clientName's document control procedures, ensuring availability, legibility, and retention as required.

\section{Standard Review}
This standard will be reviewed at least every two years or as necessitated by changes in policy, legislation, or lessons learned.

% Signature Page
\section{Signature Page}
\begin{itemize}
  \item \textbf{Compiler}: \compilerName \\
    Signature: \rule{5cm}{0.4pt} \quad Date: \rule{3cm}{0.4pt}
  \item \textbf{Approved By}: {{CEO_NAME}} \\
    Signature: \rule{5cm}{0.4pt} \quad Date: \rule{3cm}{0.4pt}
\end{itemize}

% Back Page
\ifthenelse{\equal{\brandingLevel}{heavy} \OR \equal{\brandingLevel}{light} \OR \equal{\brandingLevel}{none}}{
  \newpage
  \section*{About SafetyHelp}
  SafetyHelp is South Africa’s leading OHS platform, offering AI-powered tools for compliance, risk management, and safety education across industries like construction, mining, education, and healthcare.

  \textbf{Comprehensive Services}: \\
  \textit{OHS Tools}: \\
  \begin{itemize}
    \item \textit{Construction Safety}: Generate H\&S specifications, plans, risk assessments, legal appointments (OHSA, COIDA compliance).
    \item \textit{Risk Management Plans}: Step-by-step wizard with ISO 45001 integration.
    \item \textit{Document Library}: Policies, procedures, templates, permits, records (downloadable PDFs).
  \end{itemize}
  \textit{Training and Education}: \\
  \begin{itemize}
    \item \textit{Training Wizard}: Tailored programs by industry, audience, location (PDF guides, online videos Q3 2025).
    \item \textit{Training Provider Directory}: Connect with accredited providers.
  \end{itemize}
  \textit{Community Safety Tools}: \\
  \begin{itemize}
    \item \textit{Incident Reporting}: Report hazards (e.g., potholes, equipment issues) with photos, location, weather data, share via email/SMS/WhatsApp/X.
    \item \textit{Safety Checklists}: For spaza shops, schools, hospitals, taverns.
  \end{itemize}
  \textit{Multilingual Accessibility}: Supports 15 languages (English, Zulu, Xhosa, Afrikaans, Mandarin, etc.) for inclusivity. \\
  \textit{User Dashboard}: Profile management, saved resources (RMPs, incident reports), supplier directory with reviews. \\
  \textit{Upcoming Features}: \\
  \begin{itemize}
    \item Online Video Training (Q3 2025)
    \item Mobile App (iOS/Android, Q4 2025)
    \item AI Chatbot Enhancements (Q2 2026)
  \end{itemize}

  \textbf{Contact}: Visit \href{https://safetyfirst.help}{safetyfirst.help} or email \href{mailto:salatiso@safetyfirst.help}{salatiso@safetyfirst.help} to learn more. Become a member to customize documents and remove branding!

  \vspace{2cm}
  \hfill \includegraphics[height=2cm]{\safetyHelpLogo}
}{}

\label{LastPage}
\end{document}
