\documentclass[11pt]{article}
\usepackage[utf8]{inputenc}
\usepackage[a4paper, margin=2.5cm]{geometry}
\usepackage{fancyhdr}
\usepackage{draftwatermark}
\usepackage{tocbibind}
\usepackage{tabularx}
\usepackage{booktabs}
\usepackage{graphicx}
\usepackage{hyperref}
\usepackage{noto}
\usepackage{ifthen}

% Branding variables
\newcommand{\brandingLevel}{light}
\newcommand{\userLogo}{}
\newcommand{\userCompany}{{{COMPANY_NAME}}}
\newcommand{\safetyHelpLogo}{/safety-plans/assets/images/logo.png}
\SetWatermarkText{SafetyHelp}
\SetWatermarkScale{2}
\SetWatermarkColor[gray]{0.9}

% Dynamic fields
\newcommand{\docTitle}{Emergency Management Policy}
\newcommand{\refNumber}{EMP-001}
\newcommand{\issueDate}{{{DATE}}}
\newcommand{\projectName}{}
\newcommand{\clientName}{{{COMPANY_NAME}}}
\newcommand{\clientContact}{{{COMPANY_CONTACT}}, {{COMPANY_EMAIL}}}
\newcommand{\compilerName}{{{COMPILED_BY_NAME}}}
\newcommand{\compilerRole}{{{COMPILED_BY_ROLE}}}
\newcommand{\compilerSACPCMP}{}
\newcommand{\compilerEmail}{{{COMPILED_BY_EMAIL}}}
\newcommand{\compilerPhone}{{{COMPILED_BY_CONTACT}}}
\newcommand{\compilerCompany}{}
\newcommand{\revision}{{{REVISION}}}
\newcommand{\reviewDate}{{{REVIEW_DATE}}}

% Header and Footer
\pagestyle{fancy}
\fancyhf{}
\ifthenelse{\equal{\brandingLevel}{heavy}}{
  \fancyhead[L]{\small \refNumber}
  \fancyhead[C]{\ifthen\isempty{\userLogo}{\textbf{\userCompany}}{\includegraphics[height=1.5cm]{\userLogo}} \quad \includegraphics[height=1cm]{\safetyHelpLogo}}
  \fancyhead[R]{\small \compilerName}
  \SetWatermarkLightness{0.7}
}{
  \fancyhead[L]{\small \refNumber}
  \fancyhead[C]{\ifthen\isempty{\userLogo}{\textbf{\userCompany}}{\includegraphics[height=1.5cm]{\userLogo}}}
  \fancyhead[R]{\small \compilerName}
}
\ifthenelse{\equal{\brandingLevel}{none}}{
  \SetWatermarkText{}
}{}
\fancyfoot[L]{\small \textit{Powered by SafetyHelp | Safety Personalized, Compliance Simplified}}
\fancyfoot[C]{\small \textit{Visit safetyfirst.help | Email: salatiso@safetyfirst.help}}
\fancyfoot[R]{\small Page \thepage\ of \pageref{LastPage}}
\renewcommand{\headrulewidth}{0.4pt}
\renewcommand{\footrulewidth}{0pt}
\fancyfoot[C]{\parbox[b]{\textwidth}{\footnotesize \textit{This document is generated by SafetyHelp for OHS compliance. Verify applicability with local regulations.}}}

\begin{document}

% Cover Page
\begin{titlepage}
  \centering
  \ifthenelse{\equal{\brandingLevel}{heavy} \OR \equal{\brandingLevel}{light}}{
    \includegraphics[height=1.5cm,keepaspectratio]{\safetyHelpLogo}\hfill
  }{}
  \vspace{2cm}
  \ifthen\isempty{\userLogo}{\textbf{\Large \userCompany}\vspace{1cm}}{\includegraphics[height=1.5cm]{\userLogo}\vspace{1cm}}
  {\Huge \textbf{\docTitle}}\vspace{1cm}
  {\Large Reference: \refNumber}\vspace{0.5cm}
  {\large Issue Date: \issueDate}\vspace{0.5cm}
  {\large Review Date: \reviewDate}\vspace{0.5cm}
  {\large Client: \clientName}\vspace{0.5cm}
  {\large \clientContact}\vspace{0.5cm}
  {\large Compiled By: \compilerName, \compilerRole}\vspace{0.5cm}
  {\large \compilerEmail, \compilerPhone}
  \vspace{2cm}
  \rule{\textwidth}{0.5pt}
\end{titlepage}

% Table of Contents
\tableofcontents
\newpage

% Introduction
\section{Introduction}
\subsection{Purpose}
The purpose of this Emergency Management Policy is to affirm \clientName's commitment to protecting the health, safety, and welfare of its employees, contractors, visitors, and the public, as well as to safeguard company assets and the environment in the event of an emergency. This policy establishes the framework for preparing for, responding to, and recovering from emergencies.

\subsection{Scope}
This policy applies to all employees, contractors, visitors, and operations at all \clientName\ facilities and work sites located at {{COMPANY_ADDRESS}}. It encompasses all activities related to emergency preparedness, response, and recovery.

% Compiled By Details
\section{Compiled By Details}
\begin{tabularx}{\textwidth}{lX}
  \toprule
  \textbf{Field} & \textbf{Details} \\
  \midrule
  Name & \compilerName \\
  Role & \compilerRole \\
  SACPCMP Number & \compilerSACPCMP \\
  Email & \compilerEmail \\
  Phone & \compilerPhone \\
  Date & \issueDate \\
  Company & \compilerCompany \\
  \bottomrule
\end{tabularx}

% Revision History
\section{Revision History}
\begin{tabularx}{\textwidth}{lXll}
  \toprule
  \textbf{Rev} & \textbf{Date} & \textbf{Changes} & \textbf{Revised By} \\
  \midrule
  \revision & \issueDate & Initial version & \compilerName \\
  \bottomrule
\end{tabularx}

% Content-Specific Sections
\section{Definitions}
\begin{itemize}
    \item \textbf{Emergency:} An unforeseen event or incident that poses an immediate threat to life, health, property, or the environment, requiring immediate action. Examples include fire, medical emergencies, hazardous material spills, natural disasters, security threats, and utility failures.
    \item \textbf{Emergency Management:} The comprehensive and coordinated effort to prevent, prepare for, respond to, and recover from emergencies.
    \item \textbf{Emergency Response Plan (ERP):} A documented plan detailing the procedures and actions to be taken in response to specific emergencies.
    \item \textbf{Evacuation:} The organized, phased, and supervised withdrawal of occupants from a dangerous or potentially dangerous area to a place of safety.
    \item \textbf{Assembly Point:} A designated safe location where evacuated personnel gather for accountability.
    \item \textbf{Emergency Response Team (ERT):} A group of trained individuals responsible for implementing the Emergency Response Plan.
\end{itemize}

\section{Policy Statement}
\clientName\ is committed to:
\begin{itemize}
    \item \textbf{Preventing} emergencies where reasonably practicable through proactive risk assessment and hazard control.
    \item \textbf{Preparing} for foreseeable emergencies by developing, implementing, and maintaining robust Emergency Response Plans (ERPs), providing necessary resources, and ensuring adequate training.
    \item \textbf{Responding} effectively and efficiently to emergencies to minimize harm to people, property, and the environment.
    \item \textbf{Recovering} from emergencies by restoring normal operations as safely and quickly as possible and learning from incidents to improve future preparedness.
    \item \textbf{Complying} with all applicable South African legislation, including the Occupational Health and Safety Act 85 of 1993 and its regulations, relevant SANS standards, and local municipal by-laws pertaining to emergency management.
    \item \textbf{Fostering} a culture of safety and emergency preparedness among all employees through awareness programs and active participation.
    \item \textbf{Ensuring} effective communication channels are established for internal and external stakeholders during an emergency.
    \item \textbf{Providing} necessary training and resources to all personnel, particularly those with specific emergency response roles.
    \item \textbf{Regularly reviewing and continually improving} our emergency management system, plans, and procedures.
\end{itemize}

\section{Roles and Responsibilities}
\begin{itemize}
    \item \textbf{Top Management (CEO, {{CEO_NAME}}):}
    \begin{itemize}
        \item Overall accountability for the implementation and effectiveness of this policy.
        \item Ensuring adequate resources are allocated for emergency management.
        \item Appointing a responsible person(s) for overseeing emergency management.
        \item Visibly demonstrating leadership and commitment to emergency preparedness.
    \end{itemize}
    \item \textbf{OHS Manager (\compilerName):}
    \begin{itemize}
        \item Developing, implementing, maintaining, and reviewing the Emergency Management System and site-specific ERPs in consultation with relevant stakeholders.
        \item Coordinating emergency preparedness activities, including training and drills.
        \item Ensuring compliance with legal and other requirements.
        \item Reporting to top management on the performance of the emergency management system.
    \end{itemize}
    \item \textbf{Managers and Supervisors:}
    \begin{itemize}
        \item Ensuring employees in their areas of responsibility are aware of and understand emergency procedures.
        \item Implementing emergency procedures within their departments/sections.
        \item Ensuring participation in training and drills.
        \item Reporting any emergency-related hazards or deficiencies.
    \end{itemize}
    \item \textbf{Emergency Response Team (ERT) Members:}
    \begin{itemize}
        \item Fulfilling their designated roles and responsibilities as outlined in the ERP during an emergency.
        \item Participating in specialized training and drills.
        \item Maintaining relevant competencies and certifications.
    \end{itemize}
    \item \textbf{All Employees and Contractors:}
    \begin{itemize}
        \item Familiarizing themselves with this policy and relevant emergency procedures.
        \item Participating in emergency training and drills.
        \item Reporting emergencies, hazards, and potential incidents promptly.
        \item Following instructions from ERT members and emergency services personnel during an emergency.
        \item Taking reasonable care for their own health and safety and that of others.
    \end{itemize}
    \item \textbf{Visitors:}
    \begin{itemize}
        \item Complying with \clientName's emergency procedures as communicated to them.
    \end{itemize}
\end{itemize}

\section{Key Principles of Emergency Management}
\clientName\ will adopt the following principles in its emergency management approach:
\begin{itemize}
    \item \textbf{Prevention:} Identifying potential hazards and implementing measures to eliminate or control them.
    \item \textbf{Preparedness:} Developing plans, training personnel, and acquiring necessary resources to effectively respond to emergencies.
    \item \textbf{Response:} Taking immediate action to save lives, protect property and the environment, and limit the impact of an emergency.
    \item \textbf{Recovery:} Implementing measures to restore normal operations and support affected individuals after an emergency.
    \item \textbf{Continual Improvement:} Regularly reviewing and updating emergency plans, procedures, and capabilities based on drills, incidents, and changes in operations or regulations.
\end{itemize}

\section{Legal Compliance}
\clientName\ will ensure that all emergency management activities comply with:
\begin{itemize}
    \item The Constitution of the Republic of South Africa, 1996.
    \item The Occupational Health and Safety Act 85 of 1993 and its applicable regulations.
    \item Relevant South African National Standards (SANS), including SANS 10400-T (Fire Protection) and SANS 45001.
    \item Applicable Municipal By-laws.
    \item Any other relevant legislation or binding agreements.
\end{itemize}

\section{Communication}
\clientName\ will establish and maintain procedures for:
\begin{itemize}
    \item Internal communication regarding emergency preparedness, response, and recovery to all employees and contractors.
    \item External communication with emergency services, regulatory authorities, neighboring facilities, and other relevant stakeholders during and after an emergency.
    \item Communicating emergency procedures and evacuation routes to visitors.
\end{itemize}

\section{Training and Awareness}
\clientName\ will:
\begin{itemize}
    \item Provide appropriate emergency preparedness and response training to all employees, tailored to their roles and responsibilities.
    \item Conduct regular emergency drills to test the effectiveness of ERPs and familiarize personnel with procedures.
    \item Promote ongoing awareness of emergency procedures and safety measures.
    \item Ensure specialized training for ERT members.
\end{itemize}

\section{Policy Review}
This Emergency Management Policy will be reviewed at least every two years or as necessitated by changes in legislation, operations, or lessons learned from incidents, drills, or audits.

% Signature Page
\section{Signature Page}
\begin{itemize}
  \item \textbf{Compiler}: \compilerName \\
    Signature: \rule{5cm}{0.4pt} \quad Date: \rule{3cm}{0.4pt}
  \item \textbf{Approved By}: {{CEO_NAME}} \\
    Signature: \rule{5cm}{0.4pt} \quad Date: \rule{3cm}{0.4pt}
\end{itemize}

% Back Page
\ifthenelse{\equal{\brandingLevel}{heavy} \OR \equal{\brandingLevel}{light} \OR \equal{\brandingLevel}{none}}{
  \newpage
  \section*{About SafetyHelp}
  SafetyHelp is South Africa’s leading OHS platform, offering AI-powered tools for compliance, risk management, and safety education across industries like construction, mining, education, and healthcare.

  \textbf{Comprehensive Services}: \\
  \textit{OHS Tools}: \\
  \begin{itemize}
    \item \textit{Construction Safety}: Generate H\&S specifications, plans, risk assessments, legal appointments (OHSA, COIDA compliance).
    \item \textit{Risk Management Plans}: Step-by-step wizard with ISO 45001 integration.
    \item \textit{Document Library}: Policies, procedures, templates, permits, records (downloadable PDFs).
  \end{itemize}
  \textit{Training and Education}: \\
  \begin{itemize}
    \item \textit{Training Wizard}: Tailored programs by industry, audience, location (PDF guides, online videos Q3 2025).
    \item \textit{Training Provider Directory}: Connect with accredited providers.
  \end{itemize}
  \textit{Community Safety Tools}: \\
  \begin{itemize}
    \item \textit{Incident Reporting}: Report hazards (e.g., potholes, equipment issues) with photos, location, weather data, share via email/SMS/WhatsApp/X.
    \item \textit{Safety Checklists}: For spaza shops, schools, hospitals, taverns.
  \end{itemize}
  \textit{Multilingual Accessibility}: Supports 15 languages (English, Zulu, Xhosa, Afrikaans, Mandarin, etc.) for inclusivity. \\
  \textit{User Dashboard}: Profile management, saved resources (RMPs, incident reports), supplier directory with reviews. \\
  \textit{Upcoming Features}: \\
  \begin{itemize}
    \item Online Video Training (Q3 2025)
    \item Mobile App (iOS/Android, Q4 2025)
    \item AI Chatbot Enhancements (Q2 2026)
  \end{itemize}

  \textbf{Contact}: Visit \href{https://safetyfirst.help}{safetyfirst.help} or email \href{mailto:salatiso@safetyfirst.help}{salatiso@safetyfirst.help} to learn more. Become a member to customize documents and remove branding!

  \vspace{2cm}
  \hfill \includegraphics[height=2cm]{\safetyHelpLogo}
}{}

\label{LastPage}
\end{document}
