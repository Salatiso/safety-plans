\documentclass[11pt]{article}
\usepackage[utf8]{inputenc}
\usepackage[a4paper, margin=2.5cm]{geometry}
\usepackage{fancyhdr}
\usepackage{draftwatermark}
\usepackage{tocbibind}
\usepackage{tabularx}
\usepackage{booktabs}
\usepackage{graphicx}
\usepackage{hyperref}
\usepackage{noto}
\usepackage{ifthen}

% Branding variables
\newcommand{\brandingLevel}{light}
\newcommand{\userLogo}{}
\newcommand{\userCompany}{{{COMPANY_NAME}}}
\newcommand{\safetyHelpLogo}{/safety-plans/assets/images/logo.png}
\SetWatermarkText{SafetyHelp}
\SetWatermarkScale{2}
\SetWatermarkColor[gray]{0.9}

% Dynamic fields
\newcommand{\docTitle}{Occupational Health and Safety (OHS) Incident Management Standard}
\newcommand{\refNumber}{OHSIMS-001}
\newcommand{\issueDate}{{{DATE}}}
\newcommand{\projectName}{}
\newcommand{\clientName}{{{COMPANY_NAME}}}
\newcommand{\clientContact}{{{COMPANY_CONTACT}}, {{COMPANY_EMAIL}}}
\newcommand{\compilerName}{{{COMPILED_BY_NAME}}}
\newcommand{\compilerRole}{{{COMPILED_BY_ROLE}}}
\newcommand{\compilerSACPCMP}{}
\newcommand{\compilerEmail}{{{COMPILED_BY_EMAIL}}}
\newcommand{\compilerPhone}{{{COMPILED_BY_CONTACT}}}
\newcommand{\compilerCompany}{}
\newcommand{\revision}{{{REVISION}}}
\newcommand{\reviewDate}{{{REVIEW_DATE}}}

% Header and Footer
\pagestyle{fancy}
\fancyhf{}
\ifthenelse{\equal{\brandingLevel}{heavy}}{
  \fancyhead[L]{\small \refNumber}
  \fancyhead[C]{\ifthen\isempty{\userLogo}{\textbf{\userCompany}}{\includegraphics[height=1.5cm]{\userLogo}} \quad \includegraphics[height=1cm]{\safetyHelpLogo}}
  \fancyhead[R]{\small \compilerName}
  \SetWatermarkLightness{0.7}
}{
  \fancyhead[L]{\small \refNumber}
  \fancyhead[C]{\ifthen\isempty{\userLogo}{\textbf{\userCompany}}{\includegraphics[height=1.5cm]{\userLogo}}}
  \fancyhead[R]{\small \compilerName}
}
\ifthenelse{\equal{\brandingLevel}{none}}{
  \SetWatermarkText{}
}{}
\fancyfoot[L]{\small \textit{Powered by SafetyHelp | Safety Personalized, Compliance Simplified}}
\fancyfoot[C]{\small \textit{Visit safetyfirst.help | Email: salatiso@safetyfirst.help}}
\fancyfoot[R]{\small Page \thepage\ of \pageref{LastPage}}
\renewcommand{\headrulewidth}{0.4pt}
\renewcommand{\footrulewidth}{0pt}
\fancyfoot[C]{\parbox[b]{\textwidth}{\footnotesize \textit{This document is generated by SafetyHelp for OHS compliance. Verify applicability with local regulations.}}}

\begin{document}

% Cover Page
\begin{titlepage}
  \centering
  \ifthenelse{\equal{\brandingLevel}{heavy} \OR \equal{\brandingLevel}{light}}{
    \includegraphics[height=1.5cm,keepaspectratio]{\safetyHelpLogo}\hfill
  }{}
  \vspace{2cm}
  \ifthen\isempty{\userLogo}{\textbf{\Large \userCompany}\vspace{1cm}}{\includegraphics[height=1.5cm]{\userLogo}\vspace{1cm}}
  {\Huge \textbf{\docTitle}}\vspace{1cm}
  {\Large Reference: \refNumber}\vspace{0.5cm}
  {\large Issue Date: \issueDate}\vspace{0.5cm}
  {\large Review Date: \reviewDate}\vspace{0.5cm}
  {\large Client: \clientName}\vspace{0.5cm}
  {\large \clientContact}\vspace{0.5cm}
  {\large Compiled By: \compilerName, \compilerRole}\vspace{0.5cm}
  {\large \compilerEmail, \compilerPhone}
  \vspace{2cm}
  \rule{\textwidth}{0.5pt}
\end{titlepage}

% Table of Contents
\tableofcontents
\newpage

% Introduction
\section{Introduction}
This Occupational Health and Safety (OHS) Incident Management Standard provides the operational framework and minimum requirements for implementing the \clientName\ OHS Incident Management Policy. It details systematic processes for reporting, recording, investigating, analyzing OHS incidents, implementing corrective and preventive actions, and ensuring compliance with statutory reporting obligations.

% Compiled By Details
\section{Compiled By Details}
\begin{tabularx}{\textwidth}{lX}
  \toprule
  \textbf{Field} & \textbf{Details} \\
  \midrule
  Name & \compilerName \\
  Role & \compilerRole \\
  SACPCMP Number & \compilerSACPCMP \\
  Email & \compilerEmail \\
  Phone & \compilerPhone \\
  Date & \issueDate \\
  Company & \compilerCompany \\
  \bottomrule
\end{tabularx}

% Revision History
\section{Revision History}
\begin{tabularx}{\textwidth}{lXll}
  \toprule
  \textbf{Rev} & \textbf{Date} & \textbf{Changes} & \textbf{Revised By} \\
  \midrule
  \revision & \issueDate & Initial version & \compilerName \\
  \bottomrule
\end{tabularx}

% Content-Specific Sections
\section{Purpose}
The purpose of this standard is to:
\begin{itemize}
    \item Define procedures for incident response, reporting, and recording.
    \item Outline the methodology for incident investigations to identify root causes.
    \item Establish requirements for corrective and preventive actions.
    \item Ensure timely statutory reporting to authorities.
    \item Facilitate learning from incidents to improve OHS performance.
\end{itemize}

\section{Scope}
This standard applies to all OHS incidents occurring within any \clientName\ workplace or arising from its work activities.

\section{Definitions}
Refer to the \clientName\ OHS Incident Management Policy for key definitions. Additional definitions include:
\begin{itemize}
    \item \textbf{Root Cause Analysis (RCA):} A method to identify the fundamental cause(s) of an incident.
    \item \textbf{Corrective Action Plan (CAP):} A documented plan to address incident findings and root causes.
\end{itemize}

\section{Legal and Standard References}
This standard is guided by:
\begin{itemize}
    \item Occupational Health and Safety Act 85 of 1993 (Sections 24, 25).
    \item General Administrative Regulations (GAR 8, 9).
    \item Compensation for Occupational Injuries and Diseases Act (COIDA) 130 of 1993.
    \item SANS 45001: Occupational Health and Safety Management Systems (Clause 10.2).
\end{itemize}

\section{OHS Incident Management Process}

\subsection{Immediate Response to an Incident}
Ensure safety by providing first aid, making the area safe, and preserving evidence without disturbing the scene unless necessary.

\subsection{Incident Reporting (Internal)}
All incidents must be reported verbally immediately and documented in writing within 24 hours using the Internal OHS Incident Report Form.

\subsection{Incident Classification and Initial Assessment}
The OHS Manager shall classify the incident, determine statutory reporting requirements, and decide the level of investigation needed.

\subsection{Incident Recording}
Incidents shall be logged in the OHS Incident Register and recorded in Annexure 1 format for reportable incidents.

\subsection{Incident Investigation}
Investigations shall identify root causes using RCA techniques, involving a team including supervisors, H\&S representatives, and investigators.

\subsection{Corrective and Preventive Actions (CAPA)}
Develop and implement a CAP to address root causes, assign responsibilities, verify effectiveness, and communicate lessons learned.

\subsection{Statutory Reporting and Notification (External)}
Report incidents to the Department of Employment and Labour (DoEL) and Compensation Commissioner as required by law, using forms like WCL1/WCL2.

\subsection{Incident Data Analysis and Review}
Analyze incident trends periodically and review findings in management and H\&S committee meetings.

\subsection{Learning from Incidents and Continual Improvement}
Share lessons learned through safety alerts and update procedures and risk assessments as needed.

\section{Training for Incident Management}
Provide awareness training for employees on reporting procedures and specific training for investigators on RCA and reporting.

\section{Standard Review}
This standard will be reviewed at least every two years or as necessitated by changes in legislation, operations, or incident trends.

% Signature Page
\section{Signature Page}
\begin{itemize}
  \item \textbf{Compiler}: \compilerName \\
    Signature: \rule{5cm}{0.4pt} \quad Date: \rule{3cm}{0.4pt}
  \item \textbf{Approved By}: {{CEO_NAME}} \\
    Signature: \rule{5cm}{0.4pt} \quad Date: \rule{3cm}{0.4pt}
\end{itemize}

% Back Page
\ifthenelse{\equal{\brandingLevel}{heavy} \OR \equal{\brandingLevel}{light} \OR \equal{\brandingLevel}{none}}{
  \newpage
  \section*{About SafetyHelp}
  SafetyHelp is South Africa’s leading OHS platform, offering AI-powered tools for compliance, risk management, and safety education across industries like construction, mining, education, and healthcare.

  \textbf{Comprehensive Services}: \\
  \textit{OHS Tools}: \\
  \begin{itemize}
    \item \textit{Construction Safety}: Generate H\&S specifications, plans, risk assessments, legal appointments (OHSA, COIDA compliance).
    \item \textit{Risk Management Plans}: Step-by-step wizard with ISO 45001 integration.
    \item \textit{Document Library}: Policies, procedures, templates, permits, records (downloadable PDFs).
  \end{itemize}
  \textit{Training and Education}: \\
  \begin{itemize}
    \item \textit{Training Wizard}: Tailored programs by industry, audience, location (PDF guides, online videos Q3 2025).
    \item \textit{Training Provider Directory}: Connect with accredited providers.
  \end{itemize}
  \textit{Community Safety Tools}: \\
  \begin{itemize}
    \item \textit{Incident Reporting}: Report hazards (e.g., potholes, equipment issues) with photos, location, weather data, share via email/SMS/WhatsApp/X.
    \item \textit{Safety Checklists}: For spaza shops, schools, hospitals, taverns.
  \end{itemize}
  \textit{Multilingual Accessibility}: Supports 15 languages (English, Zulu, Xhosa, Afrikaans, Mandarin, etc.) for inclusivity. \\
  \textit{User Dashboard}: Profile management, saved resources (RMPs, incident reports), supplier directory with reviews. \\
  \textit{Upcoming Features}: \\
  \begin{itemize}
    \item Online Video Training (Q3 2025)
    \item Mobile App (iOS/Android, Q4 2025)
    \item AI Chatbot Enhancements (Q2 2026)
  \end{itemize}

  \textbf{Contact}: Visit \href{https://safetyfirst.help}{safetyfirst.help} or email \href{mailto:salatiso@safetyfirst.help}{salatiso@safetyfirst.help} to learn more. Become a member to customize documents and remove branding!

  \vspace{2cm}
  \hfill \includegraphics[height=2cm]{\safetyHelpLogo}
}{}

\label{LastPage}
\end{document}
