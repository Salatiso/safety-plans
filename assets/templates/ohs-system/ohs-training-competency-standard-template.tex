\documentclass[12pt]{article}
\usepackage[utf8]{inputenc}
\usepackage[T1]{fontenc}
\usepackage{geometry}
\geometry{a4paper, margin=1in}
\usepackage{enumitem}
\usepackage{booktabs}
\usepackage{hyperref}
\usepackage{titlesec}
\usepackage{noto}

\titleformat{\section}{\large\bfseries}{\thesection}{1em}{}
\titleformat{\subsection}{\normalsize\bfseries}{\thesubsection}{1em}{}
\titleformat{\subsubsection}{\normalsize\itshape}{\thesubsubsection}{1em}{}

\begin{document}

\begin{titlepage}
    \centering
    \vspace*{2cm}
    {\LARGE\bfseries Occupational Health and Safety (OHS) Training and Competency Standard for {{COMPANY_NAME}}\par}
    \vspace{1cm}
    {\large\itshape Standard Number: OHSTCS-001\par}
    \vspace{0.5cm}
    {\normalsize Version: {{REVISION}}\par}
    \vspace{0.5cm}
    {\normalsize Effective Date: {{DATE}}\par}
    \vspace{0.5cm}
    {\normalsize Review Date: {{REVIEW_DATE}}\par}
    \vspace{2cm}
    {\normalsize Approved by: {{CEO_NAME}}, CEO\par}
\end{titlepage}

\section{Introduction}
This Occupational Health and Safety (OHS) Training and Competency Standard provides the operational framework and minimum requirements for implementing the {{COMPANY_NAME}} OHS Training and Competency Policy. It outlines systematic processes for identifying training needs, planning and delivering OHS training, assessing and verifying competence, maintaining records, and evaluating training effectiveness.

\section{Purpose}
The purpose of this standard is to:
\begin{itemize}
    \item Define the methodology for conducting Training Needs Analysis (TNA).
    \item Establish procedures for planning, developing, and delivering OHS training.
    \item Outline methods for assessing and verifying competence.
    \item Ensure consistency in training management.
    \item Support compliance with legal and other requirements.
    \item Facilitate continual improvement of training programs.
\end{itemize}

\section{Scope}
This standard applies to all OHS training and competency management activities for {{COMPANY_NAME}} employees and contractors performing work on behalf of the company.

\section{Definitions}
Refer to the {{COMPANY_NAME}} OHS Training and Competency Policy for key definitions. Additional definitions include:
\begin{itemize}
    \item \textbf{Training Matrix:} A document mapping required OHS training for different roles.
    \item \textbf{Competency Assessment:} The process of collecting evidence to judge competence.
\end{itemize}

\section{Legal and Standard References}
This standard is guided by:
\begin{itemize}
    \item Occupational Health and Safety Act 85 of 1993 (Sections 8, 13).
    \item Construction Regulations, 2014.
    \item General Safety Regulations.
    \item Basic Conditions of Employment Act 75 of 1997.
    \item SANS 45001: Occupational Health and Safety Management Systems (Clauses 7.2, 7.3).
\end{itemize}

\section{OHS Training and Competency Management Process}

\subsection{Training Needs Analysis (TNA)}
Line Managers shall conduct TNA by analyzing job roles, risk assessments, legal requirements, and incident reports to identify training needs.

\subsection{Training Planning and Scheduling}
An annual OHS Training Plan/Matrix shall be developed, specifying training types, target audiences, schedules, and budgets.

\subsection{Development and Sourcing of Training}
Training materials may be developed internally or sourced from accredited external providers, ensuring relevance and quality.

\subsection{Delivery of Training}
Training includes OHS induction, task-specific, hazard-specific, and refresher training, delivered in accessible formats considering language and literacy.

\subsection{Assessment of Competence and Training Effectiveness}
Competence shall be assessed through tests, observations, or certifications. Training effectiveness shall be evaluated via feedback and performance reviews.

\subsection{Record Keeping}
Comprehensive training and competency records shall be maintained, including trainee details, course content, and assessment results.

\subsection{OHS Awareness}
Promote OHS awareness through communications, campaigns, signage, and meetings.

\section{Review of Training and Competency Program}
The training program shall be reviewed annually, considering effectiveness, legal changes, and feedback.

\section{Standard Review}
This standard will be reviewed at least every two years or as necessitated by changes in legislation, operations, or training evaluations.

\section{Signature}
\textbf{Approved By:}

{{CEO_NAME}}\\
\textbf{CEO, {{COMPANY_NAME}}}\\
\textbf{Date:} {{DATE}}

\section{Contact Information}
For inquiries, contact:
\begin{itemize}
    \item Phone: {{COMPANY_CONTACT}}
    \item Email: {{COMPANY_EMAIL}}
\end{itemize}

\section{Document Control}
\begin{table}[h]
    \centering
    \begin{tabular}{p{3cm}p{3cm}p{6cm}}
        \toprule
        \textbf{Version} & \textbf{Date} & \textbf{Changes} \\
        \midrule
        {{REVISION}} & {{DATE}} & Initial version \\
        \bottomrule
    \end{tabular}
    \caption{Revision History}
\end{table}

\section{Compiled By}
\begin{itemize}
    \item \textbf{Name}: {{COMPILED_BY_NAME}}
    \item \textbf{Contact}: {{COMPILED_BY_CONTACT}}
    \item \textbf{Email}: {{COMPILED_BY_EMAIL}}
    \item \textbf{Role}: {{COMPILED_BY_ROLE}}
\end{itemize}

\end{document}
