\documentclass[11pt]{article}
\usepackage[utf8]{inputenc}
\usepackage[a4paper, margin=2.5cm]{geometry}
\usepackage{fancyhdr}
\usepackage{draftwatermark}
\usepackage{tocbibind}
\usepackage{tabularx}
\usepackage{booktabs}
\usepackage{graphicx}
\usepackage{hyperref}
\usepackage{noto}
\usepackage{ifthen}

% Branding variables
\newcommand{\brandingLevel}{light}
\newcommand{\userLogo}{}
\newcommand{\userCompany}{{{COMPANY_NAME}}}
\newcommand{\safetyHelpLogo}{/safety-plans/assets/images/logo.png}
\SetWatermarkText{SafetyHelp}
\SetWatermarkScale{2}
\SetWatermarkColor[gray]{0.9}

% Dynamic fields
\newcommand{\docTitle}{Occupational Health and Safety (OHS) Training and Competency Standard}
\newcommand{\refNumber}{OHSTCS-001}
\newcommand{\issueDate}{{{DATE}}}
\newcommand{\projectName}{}
\newcommand{\clientName}{{{COMPANY_NAME}}}
\newcommand{\clientContact}{{{COMPANY_CONTACT}}, {{COMPANY_EMAIL}}}
\newcommand{\compilerName}{{{COMPILED_BY_NAME}}}
\newcommand{\compilerRole}{{{COMPILED_BY_ROLE}}}
\newcommand{\compilerSACPCMP}{}
\newcommand{\compilerEmail}{{{COMPILED_BY_EMAIL}}}
\newcommand{\compilerPhone}{{{COMPILED_BY_CONTACT}}}
\newcommand{\compilerCompany}{}
\newcommand{\revision}{{{REVISION}}}
\newcommand{\reviewDate}{{{REVIEW_DATE}}}

% Header and Footer
\pagestyle{fancy}
\fancyhf{}
\ifthenelse{\equal{\brandingLevel}{heavy}}{
  \fancyhead[L]{\small \refNumber}
  \fancyhead[C]{\ifthen\isempty{\userLogo}{\textbf{\userCompany}}{\includegraphics[height=1.5cm]{\userLogo}} \quad \includegraphics[height=1cm]{\safetyHelpLogo}}
  \fancyhead[R]{\small \compilerName}
  \SetWatermarkLightness{0.7}
}{
  \fancyhead[L]{\small \refNumber}
  \fancyhead[C]{\ifthen\isempty{\userLogo}{\textbf{\userCompany}}{\includegraphics[height=1.5cm]{\userLogo}}}
  \fancyhead[R]{\small \compilerName}
}
\ifthenelse{\equal{\brandingLevel}{none}}{
  \SetWatermarkText{}
}{}
\fancyfoot[L]{\small \textit{Powered by SafetyHelp | Safety Personalized, Compliance Simplified}}
\fancyfoot[C]{\small \textit{Visit safetyfirst.help | Email: salatiso@safetyfirst.help}}
\fancyfoot[R]{\small Page \thepage\ of \pageref{LastPage}}
\renewcommand{\headrulewidth}{0.4pt}
\renewcommand{\footrulewidth}{0pt}
\fancyfoot[C]{\parbox[b]{\textwidth}{\footnotesize \textit{This document is generated by SafetyHelp for OHS compliance. Verify applicability with local regulations.}}}

\begin{document}

% Cover Page
\begin{titlepage}
  \centering
  \ifthenelse{\equal{\brandingLevel}{heavy} \OR \equal{\brandingLevel}{light}}{
    \includegraphics[height=1.5cm,keepaspectratio]{\safetyHelpLogo}\hfill
  }{}
  \vspace{2cm}
  \ifthen\isempty{\userLogo}{\textbf{\Large \userCompany}\vspace{1cm}}{\includegraphics[height=1.5cm]{\userLogo}\vspace{1cm}}
  {\Huge \textbf{\docTitle}}\vspace{1cm}
  {\Large Reference: \refNumber}\vspace{0.5cm}
  {\large Issue Date: \issueDate}\vspace{0.5cm}
  {\large Review Date: \reviewDate}\vspace{0.5cm}
  {\large Client: \clientName}\vspace{0.5cm}
  {\large \clientContact}\vspace{0.5cm}
  {\large Compiled By: \compilerName, \compilerRole}\vspace{0.5cm}
  {\large \compilerEmail, \compilerPhone}
  \vspace{2cm}
  \rule{\textwidth}{0.5pt}
\end{titlepage}

% Table of Contents
\tableofcontents
\newpage

% Introduction
\section{Introduction}
This Occupational Health and Safety (OHS) Training and Competency Standard provides the operational framework and minimum requirements for implementing the \clientName\ OHS Training and Competency Policy. It outlines systematic processes for identifying training needs, planning and delivering OHS training, assessing and verifying competence, maintaining records, and evaluating training effectiveness.

% Compiled By Details
\section{Compiled By Details}
\begin{tabularx}{\textwidth}{lX}
  \toprule
  \textbf{Field} & \textbf{Details} \\
  \midrule
  Name & \compilerName \\
  Role & \compilerRole \\
  SACPCMP Number & \compilerSACPCMP \\
  Email & \compilerEmail \\
  Phone & \compilerPhone \\
  Date & \issueDate \\
  Company & \compilerCompany \\
  \bottomrule
\end{tabularx}

% Revision History
\section{Revision History}
\begin{tabularx}{\textwidth}{lXll}
  \toprule
  \textbf{Rev} & \textbf{Date} & \textbf{Changes} & \textbf{Revised By} \\
  \midrule
  \revision & \issueDate & Initial version & \compilerName \\
  \bottomrule
\end{tabularx}

% Content-Specific Sections
\section{Purpose}
The purpose of this standard is to:
\begin{itemize}
    \item Define the methodology for conducting Training Needs Analysis (TNA).
    \item Establish procedures for planning, developing, and delivering OHS training.
    \item Outline methods for assessing and verifying competence.
    \item Ensure consistency in training management.
    \item Support compliance with legal and other requirements.
    \item Facilitate continual improvement of training programs.
\end{itemize}

\section{Scope}
This standard applies to all OHS training and competency management activities for \clientName\ employees and contractors performing work on behalf of the company.

\section{Definitions}
Refer to the \clientName\ OHS Training and Competency Policy for key definitions. Additional definitions include:
\begin{itemize}
    \item \textbf{Training Matrix:} A document mapping required OHS training for different roles.
    \item \textbf{Competency Assessment:} The process of collecting evidence to judge competence.
\end{itemize}

\section{Legal and Standard References}
This standard is guided by:
\begin{itemize}
    \item Occupational Health and Safety Act 85 of 1993 (Sections 8, 13).
    \item Construction Regulations, 2014.
    \item General Safety Regulations.
    \item Basic Conditions of Employment Act 75 of 1997.
    \item SANS 45001: Occupational Health and Safety Management Systems (Clauses 7.2, 7.3).
\end{itemize}

\section{OHS Training and Competency Management Process}

\subsection{Training Needs Analysis (TNA)}
Line Managers shall conduct TNA by analyzing job roles, risk assessments, legal requirements, and incident reports to identify training needs.

\subsection{Training Planning and Scheduling}
An annual OHS Training Plan/Matrix shall be developed, specifying training types, target audiences, schedules, and budgets.

\subsection{Development and Sourcing of Training}
Training materials may be developed internally or sourced from accredited external providers, ensuring relevance and quality.

\subsection{Delivery of Training}
Training includes OHS induction, task-specific, hazard-specific, and refresher training, delivered in accessible formats considering language and literacy.

\subsection{Assessment of Competence and Training Effectiveness}
Competence shall be assessed through tests, observations, or certifications. Training effectiveness shall be evaluated via feedback and performance reviews.

\subsection{Record Keeping}
Comprehensive training and competency records shall be maintained, including trainee details, course content, and assessment results.

\subsection{OHS Awareness}
Promote OHS awareness through communications, campaigns, signage, and meetings.

\section{Review of Training and Competency Program}
The training program shall be reviewed annually, considering effectiveness, legal changes, and feedback.

\section{Standard Review}
This standard will be reviewed at least every two years or as necessitated by changes in legislation, operations, or training evaluations.

% Signature Page
\section{Signature Page}
\begin{itemize}
  \item \textbf{Compiler}: \compilerName \\
    Signature: \rule{5cm}{0.4pt} \quad Date: \rule{3cm}{0.4pt}
  \item \textbf{Approved By}: {{CEO_NAME}} \\
    Signature: \rule{5cm}{0.4pt} \quad Date: \rule{3cm}{0.4pt}
\end{itemize}

% Back Page
\ifthenelse{\equal{\brandingLevel}{heavy} \OR \equal{\brandingLevel}{light} \OR \equal{\brandingLevel}{none}}{
  \newpage
  \section*{About SafetyHelp}
  SafetyHelp is South Africa’s leading OHS platform, offering AI-powered tools for compliance, risk management, and safety education across industries like construction, mining, education, and healthcare.

  \textbf{Comprehensive Services}: \\
  \textit{OHS Tools}: \\
  \begin{itemize}
    \item \textit{Construction Safety}: Generate H\&S specifications, plans, risk assessments, legal appointments (OHSA, COIDA compliance).
    \item \textit{Risk Management Plans}: Step-by-step wizard with ISO 45001 integration.
    \item \textit{Document Library}: Policies, procedures, templates, permits, records (downloadable PDFs).
  \end{itemize}
  \textit{Training and Education}: \\
  \begin{itemize}
    \item \textit{Training Wizard}: Tailored programs by industry, audience, location (PDF guides, online videos Q3 2025).
    \item \textit{Training Provider Directory}: Connect with accredited providers.
  \end{itemize}
  \textit{Community Safety Tools}: \\
  \begin{itemize}
    \item \textit{Incident Reporting}: Report hazards (e.g., potholes, equipment issues) with photos, location, weather data, share via email/SMS/WhatsApp/X.
    \item \textit{Safety Checklists}: For spaza shops, schools, hospitals, taverns.
  \end{itemize}
  \textit{Multilingual Accessibility}: Supports 15 languages (English, Zulu, Xhosa, Afrikaans, Mandarin, etc.) for inclusivity. \\
  \textit{User Dashboard}: Profile management, saved resources (RMPs, incident reports), supplier directory with reviews. \\
  \textit{Upcoming Features}: \\
  \begin{itemize}
    \item Online Video Training (Q3 2025)
    \item Mobile App (iOS/Android, Q4 2025)
    \item AI Chatbot Enhancements (Q2 2026)
  \end{itemize}

  \textbf{Contact}: Visit \href{https://safetyfirst.help}{safetyfirst.help} or email \href{mailto:salatiso@safetyfirst.help}{salatiso@safetyfirst.help} to learn more. Become a member to customize documents and remove branding!

  \vspace{2cm}
  \hfill \includegraphics[height=2cm]{\safetyHelpLogo}
}{}

\label{LastPage}
\end{document}
