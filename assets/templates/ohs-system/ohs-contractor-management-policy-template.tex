\documentclass[12pt]{article}
\usepackage[utf8]{inputenc}
\usepackage[T1]{fontenc}
\usepackage{geometry}
\geometry{a4paper, margin=1in}
\usepackage{enumitem}
\usepackage{booktabs}
\usepackage{hyperref}
\usepackage{titlesec}
\usepackage{noto}

\titleformat{\section}{\large\bfseries}{\thesection}{1em}{}
\titleformat{\subsection}{\normalsize\bfseries}{\thesubsection}{1em}{}
\titleformat{\subsubsection}{\normalsize\itshape}{\thesubsubsection}{1em}{}

\begin{document}

\begin{titlepage}
    \centering
    \vspace*{2cm}
    {\LARGE\bfseries Occupational Health and Safety (OHS) Contractor Management Policy for {{COMPANY_NAME}}\par}
    \vspace{1cm}
    {\large\itshape Policy Number: OHSCMP-001\par}
    \vspace{0.5cm}
    {\normalsize Version: {{REVISION}}\par}
    \vspace{0.5cm}
    {\normalsize Effective Date: {{DATE}}\par}
    \vspace{0.5cm}
    {\normalsize Review Date: {{REVIEW_DATE}}\par}
    \vspace{2cm}
    {\normalsize Approved by: {{CEO_NAME}}, CEO\par}
\end{titlepage}

\section{Purpose}
The purpose of this Occupational Health and Safety (OHS) Contractor Management Policy is to affirm {{COMPANY_NAME}}'s commitment to ensuring that all contractors engaged in work on behalf of the company perform their activities in a manner that prioritizes OHS. This policy establishes the framework for selecting, evaluating, monitoring, and managing contractors to ensure compliance with OHS legal requirements, company standards, and contractual obligations, thereby protecting employees, contractors, and the public.

\section{Scope}
This policy applies to all contractors, subcontractors, and service providers engaged by {{COMPANY_NAME}} for work at company sites, projects, or facilities in South Africa. It covers the entire contractor lifecycle, from pre-qualification and selection to performance monitoring, incident management, and contract close-out.

\section{Definitions}
\begin{itemize}
    \item \textbf{Contractor:} Any individual, company, or entity engaged by {{COMPANY_NAME}} to perform work or provide services, including subcontractors.
    \item \textbf{OHS Management System:} The framework of policies, procedures, and resources for managing OHS risks, as aligned with SANS 45001.
    \item \textbf{Health and Safety File:} A documented record of a contractor’s OHS management system, plans, and compliance, as required by Construction Regulations 2014.
    \item \textbf{Contractual OHS Penalties:} Financial or contractual consequences linked to the main project contract for non-compliance with OHS requirements.
\end{itemize}

\section{Policy Statement}
{{COMPANY_NAME}} is committed to:
\begin{itemize}
    \item \textbf{Contractor Selection and Evaluation:} Selecting contractors based on their OHS performance, incident records, and the robustness of their OHS management systems.
    \item \textbf{Legal Compliance:} Ensuring contractors comply with South African OHS legislation, including the Occupational Health and Safety Act 85 of 1993, Construction Regulations 2014, and other applicable regulations.
    \item \textbf{Contractual Obligations:} Incorporating OHS requirements into contracts, aligning with standards such as JBCC, NEC, FIDIC, and GCC, and linking OHS penalties to the main project contract for non-compliance.
    \item \textbf{Pre-Qualification and Due Diligence:} Conducting pre-qualification assessments to verify contractors’ OHS capabilities and compliance before engagement.
    \item \textbf{Induction and Training:} Providing OHS induction and necessary training to contractors to ensure awareness of site-specific hazards and company procedures.
    \item \textbf{Monitoring and Supervision:} Actively monitoring contractor OHS performance through inspections, audits, and incident reporting.
    \item \textbf{Incident Management:} Requiring contractors to report incidents, participate in investigations, and implement corrective actions, with incident data used in ongoing evaluations.
    \item \textbf{Performance Evaluation:} Regularly evaluating contractors’ OHS performance, using metrics such as incident rates, compliance with safety plans, and audit findings.
    \item \textbf{Continuous Improvement:} Promoting continual improvement in contractor OHS performance through feedback, lessons learned, and shared best practices.
    \item \textbf{Resource Support:} Providing contractors with necessary support, guidance, and resources to meet OHS requirements.
\end{itemize}

\section{Roles and Responsibilities}
\begin{itemize}
    \item \textbf{Top Management (CEO, {{CEO_NAME}}):}
    \begin{itemize}
        \item Overall accountability for ensuring effective contractor OHS management.
        \item Approving policies and ensuring resources for implementation.
    \end{itemize}
    \item \textbf{OHS Manager ({{COMPILED_BY_NAME}}):}
    \begin{itemize}
        \item Developing and overseeing the contractor management framework.
        \item Conducting pre-qualification and performance evaluations.
        \item Ensuring compliance with legal and contractual OHS requirements.
    \end{itemize}
    \item \textbf{Project Managers:}
    \begin{itemize}
        \item Incorporating OHS requirements into contracts and monitoring compliance.
        \item Enforcing contractual OHS penalties as per the main project contract.
        \item Coordinating with contractors on OHS matters.
    \end{itemize}
    \item \textbf{Contractors:}
    \begin{itemize}
        \item Complying with all OHS legal, contractual, and company requirements.
        \item Maintaining and implementing their own OHS management systems.
        \item Reporting incidents and participating in investigations.
    \end{itemize}
\end{itemize}

\section{Key Principles for OHS Contractor Management}
\begin{itemize}
    \item \textbf{Risk-Based Approach:} Prioritizing OHS management based on the risks associated with contractor activities.
    \item \textbf{Accountability:} Holding contractors accountable for OHS performance through contractual mechanisms.
    \item \textbf{Transparency:} Ensuring clear communication of OHS expectations and performance feedback.
    \item \textbf{Integration:} Embedding OHS considerations into all stages of contractor engagement.
\end{itemize}

\section{Legal and Standard Framework}
This policy is guided by:
\begin{itemize}
    \item Occupational Health and Safety Act 85 of 1993 (Sections 8, 9, 37).
    \item Construction Regulations, 2014 (CR 5: Duties of Client, CR 7: Duties of Contractors).
    \item SANS 45001: Occupational Health and Safety Management Systems (Clause 8.1.6: Contractors).
    \item JBCC, NEC, FIDIC, and GCC contract standards (e.g., JBCC Clause 12: Safety; NEC Clause 27: Health and Safety; FIDIC Clause 4.8: Safety Procedures; GCC Clause 4.9: Contractor’s Responsibilities).
\end{itemize}

\section{Policy Review}
This OHS Contractor Management Policy will be reviewed at least every two years or as necessitated by changes in legislation, operations, or contractor performance trends.

\section{Signature}
\textbf{Signed:}

{{CEO_NAME}}\\
\textbf{CEO, {{COMPANY_NAME}}}\\
\textbf{Date:} {{DATE}}

\section{Contact Information}
For inquiries, contact:
\begin{itemize}
    \item Phone: {{COMPANY_CONTACT}}
    \item Email: {{COMPANY_EMAIL}}
\end{itemize}

\section{Document Control}
\begin{table}[h]
    \centering
    \begin{tabular}{p{3cm}p{3cm}p{6cm}}
        \toprule
        \textbf{Version} & \textbf{Date} & \textbf{Changes} \\
        \midrule
        {{REVISION}} & {{DATE}} & Initial version \\
        \bottomrule
    \end{tabular}
    \caption{Revision History}
\end{table}

\section{Compiled By}
\begin{itemize}
    \item \textbf{Name}: {{COMPILED_BY_NAME}}
    \item \textbf{Contact}: {{COMPILED_BY_CONTACT}}
    \item \textbf{Email}: {{COMPILED_BY_EMAIL}}
    \item \textbf{Role}: {{COMPILED_BY_ROLE}}
\end{itemize}

\end{document}
