\documentclass[12pt]{article}
\usepackage[utf8]{inputenc}
\usepackage[T1]{fontenc}
\usepackage{geometry}
\geometry{a4paper, margin=1in}
\usepackage{enumitem}
\usepackage{booktabs}
\usepackage{hyperref}
\usepackage{titlesec}
\usepackage{noto}

\titleformat{\section}{\large\bfseries}{\thesection}{1em}{}
\titleformat{\subsection}{\normalsize\bfseries}{\thesubsection}{1em}{}
\titleformat{\subsubsection}{\normalsize\itshape}{\thesubsubsection}{1em}{}

\begin{document}

\begin{titlepage}
    \centering
    \vspace*{2cm}
    {\LARGE\bfseries Occupational Health and Safety (OHS) Documentation and Records Control Policy for {{COMPANY_NAME}}\par}
    \vspace{1cm}
    {\large\itshape Policy Number: OHSDRC-001\par}
    \vspace{0.5cm}
    {\normalsize Version: {{REVISION}}\par}
    \vspace{0.5cm}
    {\normalsize Effective Date: {{DATE}}\par}
    \vspace{0.5cm}
    {\normalsize Review Date: {{REVIEW_DATE}}\par}
    \vspace{2cm}
    {\normalsize Approved by: {{CEO_NAME}}, CEO\par}
\end{titlepage}

\section{Purpose}
The purpose of this Occupational Health and Safety (OHS) Documentation and Records Control Policy is to affirm {{COMPANY_NAME}}'s commitment to systematically managing OHS documentation and records. This policy ensures that all OHS-related documents and records are created, controlled, retained, and accessible to support compliance, audits, and continual improvement of the OHS Management System.

\section{Scope}
This policy applies to all OHS documentation and records generated or maintained by {{COMPANY_NAME}} across its operations in South Africa, including policies, procedures, risk assessments, incident reports, training records, and audit findings.

\section{Definitions}
\begin{itemize}
    \item \textbf{OHS Documentation:} Written materials (e.g., policies, procedures, plans) that define the OHS Management System.
    \item \textbf{OHS Records:} Evidence of activities performed or results achieved (e.g., incident reports, training logs).
    \item \textbf{Document Control:} The process of managing the creation, review, approval, distribution, and revision of documents.
    \item \textbf{Records Retention:} The systematic retention and disposal of records per legal and organizational requirements.
\end{itemize}

\section{Policy Statement}
{{COMPANY_NAME}} is committed to:
\begin{itemize}
    \item \textbf{Document Control:} Establishing processes to create, review, approve, and distribute OHS documentation, ensuring accuracy and accessibility.
    \item \textbf{Records Management:} Maintaining OHS records systematically to provide evidence of compliance and performance.
    \item \textbf{Legal Compliance:} Complying with South African legislation (e.g., OHS Act Section 7.5, General Administrative Regulations) and SANS 45001 (Clause 7.5).
    \item \textbf{Accessibility:} Ensuring OHS documentation and records are accessible to relevant personnel while maintaining confidentiality where required.
    \item \textbf{Retention and Disposal:} Defining retention periods for OHS records and ensuring secure disposal after the retention period.
    \item \textbf{Version Control:} Implementing version control to ensure the use of current and approved OHS documents.
    \item \textbf{Training:** Providing training on document and records control processes to relevant employees.
    \item \textbf{Continual Improvement:** Regularly reviewing and improving documentation and records management practices.
\end{itemize}

\section{Roles and Responsibilities}
\begin{itemize}
    \item \textbf{Top Management (CEO, {{CEO_NAME}}):}
    \begin{itemize}
        \item Ensuring resources for effective documentation and records control.
        \item Approving high-level OHS documentation.
    \end{itemize}
    \item \textbf{OHS Manager ({{COMPILED_BY_NAME}}):}
    \begin{itemize}
        \item Overseeing the development and maintenance of OHS documentation and records.
        \item Managing the document control and records retention processes.
    \end{itemize}
    \item \textbf{Line Managers:}
    \begin{itemize}
        \item Ensuring compliance with document control procedures in their areas.
        \item Maintaining accurate records of OHS activities.
    \end{itemize}
    \item \textbf{Employees:}
    \begin{itemize}
        \item Using approved OHS documentation and following records management procedures.
        \item Reporting outdated or missing documents/records.
    \end{itemize}
\end{itemize}

\section{Key Principles}
\begin{itemize}
    \item \textbf{Accuracy:** Ensuring all OHS documentation and records are accurate and up-to-date.
    \item \textbf{Traceability:** Maintaining records to provide a clear audit trail.
    \item \textbf{Security:** Protecting sensitive OHS records from unauthorized access.
\end{itemize}

\section{Legal and Standard Framework}
This policy is guided by:
\begin{itemize}
    \item Occupational Health and Safety Act 85 of 1993 (Section 7.5: Documented Information).
    \item General Administrative Regulations (GAR 9: Recording of Incidents).
    \item SANS 45001: Occupational Health and Safety Management Systems (Clause 7.5).
\end{itemize}

\section{Policy Review}
This policy will be reviewed at least every two years or as necessitated by changes in legislation, operations, or system requirements.

\section{Signature}
\textbf{Signed:}

{{CEO_NAME}}\\
\textbf{CEO, {{COMPANY_NAME}}}\\
\textbf{Date:} {{DATE}}

\section{Contact Information}
For inquiries, contact:
\begin{itemize}
    \item Phone: {{COMPANY_CONTACT}}
    \item Email: {{COMPANY_EMAIL}}
\end{itemize}

\section{Document Control}
\begin{table}[h]
    \centering
    \begin{tabular}{p{3cm}p{3cm}p{6cm}}
        \toprule
        \textbf{Version} & \textbf{Date} & \textbf{Changes} \\
        \midrule
        {{REVISION}} & {{DATE}} & Initial version \\
        \bottomrule
    \end{tabular}
    \caption{Revision History}
\end{table}

\section{Compiled By}
\begin{itemize}
    \item \textbf{Name}: {{COMPILED_BY_NAME}}
    \item \textbf{Contact}: {{COMPILED_BY_CONTACT}}
    \item \textbf{Email}: {{COMPILED_BY_EMAIL}}
    \item \textbf{Role}: {{COMPILED_BY_ROLE}}
\end{itemize}

\end{document}
