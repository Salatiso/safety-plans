\documentclass[12pt]{article}
\usepackage[utf8]{inputenc}
\usepackage[T1]{fontenc}
\usepackage{geometry}
\geometry{a4paper, margin=1in}
\usepackage{enumitem}
\usepackage{booktabs}
\usepackage{hyperref}
\usepackage{titlesec}
\usepackage{noto}

\titleformat{\section}{\large\bfseries}{\thesection}{1em}{}
\titleformat{\subsection}{\normalsize\bfseries}{\thesubsection}{1em}{}
\titleformat{\subsubsection}{\normalsize\itshape}{\thesubsubsection}{1em}{}

\begin{document}

\begin{titlepage}
    \centering
    \vspace*{2cm}
    {\LARGE\bfseries Construction Occupational Health and Safety (OHS) Policy for {{COMPANY_NAME}}\par}
    \vspace{1cm}
    {\large\itshape Policy Number: COHS-001\par}
    \vspace{0.5cm}
    {\normalsize Version: {{REVISION}}\par}
    \vspace{0.5cm}
    {\normalsize Effective Date: {{DATE}}\par}
    \vspace{0.5cm}
    {\normalsize Review Date: {{REVIEW_DATE}}\par}
    \vspace{2cm}
    {\normalsize Approved by: {{CEO_NAME}}, CEO\par}
\end{titlepage}

\section{Purpose}
The purpose of this Construction Occupational Health and Safety (OHS) Policy is to affirm {{COMPANY_NAME}}'s unwavering commitment to establishing and maintaining the highest practicable standards of health and safety for all persons involved in or affected by our construction activities. This policy provides the framework for managing OHS risks inherent in construction work and ensuring compliance with all applicable legal and other requirements.

\section{Scope}
This policy applies to all construction projects undertaken by {{COMPANY_NAME}}, including all employees, contractors, subcontractors, suppliers, clients, and visitors involved in or present at our construction sites within South Africa. It covers all phases of construction projects, from design and planning through execution to commissioning and handover.

\section{Definitions}
\begin{itemize}
    \item \textbf{Construction Work:} As defined in the Construction Regulations, 2014, under the Occupational Health and Safety Act 85 of 1993.
    \item \textbf{Hazard:} A source of potential harm or a situation with a potential to cause injury or ill health.
    \item \textbf{Risk:} The combination of the likelihood of an occurrence of a hazardous event or exposure(s) and the severity of injury or ill health that can be caused by the event or exposure(s).
    \item \textbf{Health and Safety (H\&S) Plan:} A documented plan, specific to a construction project, which addresses the identified hazards and risks and outlines the systems and procedures to ensure health and safety.
    \item \textbf{Competent Person:} A person who has the required knowledge, training, experience, and, where applicable, qualifications, specific to the work or task being performed.
\end{itemize}

\section{Policy Statement}
{{COMPANY_NAME}} is unequivocally committed to:
\begin{itemize}
    \item \textbf{Prioritizing Safety:} Ensuring that health and safety are integral to all our construction planning, operations, and decision-making processes. We believe that all work-related injuries, illnesses, and incidents are preventable.
    \item \textbf{Legal Compliance:} Complying with, and where reasonably practicable exceeding, the requirements of the South African Occupational Health and Safety Act 85 of 1993 and its regulations (especially the Construction Regulations, 2014), the Compensation for Occupational Injuries and Diseases Act (COIDA), relevant SANS standards (including SANS 45001 and SANS 10400), and applicable municipal by-laws.
    \item \textbf{Risk Management:} Systematically identifying construction-related hazards, assessing risks, and implementing effective control measures based on the hierarchy of controls to prevent incidents, injuries, and ill health.
    \item \textbf{Competency and Training:} Ensuring that all employees and contractors are competent for the tasks they perform through appropriate information, instruction, training, and supervision.
    \item \textbf{Consultation and Participation:} Promoting active consultation and participation of employees, contractors, and their representatives in OHS matters.
    \item \textbf{Resource Allocation:} Providing adequate resources to implement this policy and achieve our OHS objectives on all construction projects.
    \item \textbf{Subcontractor Management:} Ensuring that all subcontractors adhere to our OHS standards and comply with all relevant legal requirements.
    \item \textbf{Emergency Preparedness:} Establishing and maintaining effective emergency response plans for all construction sites.
    \item \textbf{Incident Management:} Reporting, recording, and investigating all OHS incidents to identify root causes and implement corrective actions.
    \item \textbf{Continual Improvement:} Regularly reviewing OHS performance, management systems, and this policy to identify opportunities for continual improvement.
    \item \textbf{Client and Designer Collaboration:} Working collaboratively with clients and designers to ensure health and safety are considered from the earliest stages of a project lifecycle.
    \item \textbf{Public Safety:} Conducting our construction activities in a manner that ensures the health and safety of members of the public.
\end{itemize}

\section{Roles and Responsibilities}
\begin{itemize}
    \item \textbf{Top Management (CEO, {{CEO_NAME}}):}
    \begin{itemize}
        \item Overall accountability for the implementation and effectiveness of this Construction OHS Policy.
        \item Providing visible leadership and commitment to OHS in construction.
        \item Ensuring adequate resources are allocated for construction OHS management.
    \end{itemize}
    \item \textbf{Construction OHS Manager ({{COMPILED_BY_NAME}}):}
    \begin{itemize}
        \item Developing, implementing, maintaining, and reviewing the Construction OHS Management System.
        \item Ensuring compliance with legal and other requirements across all construction projects.
        \item Coordinating OHS training, audits, and inspections for construction activities.
    \end{itemize}
    \item \textbf{Project/Site Management:}
    \begin{itemize}
        \item Implementing and enforcing OHS requirements on their specific construction sites.
        \item Ensuring a project-specific H\&S Plan is developed, implemented, and maintained.
        \item Conducting regular site inspections and ensuring hazards are identified and controlled.
    \end{itemize}
    \item \textbf{All Employees and Contractors:}
    \begin{itemize}
        \item Complying with this policy, site-specific H\&S Plans, and legal OHS requirements.
        \item Reporting any hazards, incidents, or unsafe conditions immediately.
        \item Participating in OHS training and inductions.
    \end{itemize}
\end{itemize}

\section{Key Principles for Construction OHS}
{{COMPANY_NAME}} will manage OHS on its construction projects based on the following principles:
\begin{itemize}
    \item \textbf{Planning for Safety:} Integrating OHS into all project planning stages.
    \item \textbf{Hazard Identification and Risk Assessment (HIRA):} Conducting comprehensive HIRAs for all construction activities.
    \item \textbf{Hierarchy of Controls:} Prioritizing control measures to eliminate hazards.
    \item \textbf{Safe Work Procedures (SWPs):} Developing and implementing SWPs for all significant and high-risk tasks.
    \item \textbf{Competency and Supervision:} Ensuring all work is carried out by competent individuals under appropriate supervision.
    \item \textbf{Communication and Consultation:} Maintaining open communication channels and consulting with all stakeholders.
    \item \textbf{Monitoring and Review:} Regularly monitoring OHS performance through inspections and audits.
\end{itemize}

\section{Legal and Regulatory Compliance}
{{COMPANY_NAME}} is committed to full compliance with:
\begin{itemize}
    \item The Occupational Health and Safety Act 85 of 1993.
    \item The Construction Regulations, 2014.
    \item The Compensation for Occupational Injuries and Diseases Act 130 of 1993 (COIDA).
    \item Relevant SANS Standards (e.g., SANS 45001, SANS 10400).
\end{itemize}

\section{Communication and Consultation}
{{COMPANY_NAME}} will establish effective communication and consultation mechanisms, including:
\begin{itemize}
    \item Site OHS inductions.
    \item Toolbox talks.
    \item Health and Safety Committee meetings.
    \item Visible display of OHS information.
\end{itemize}

\section{Training and Competence}
{{COMPANY_NAME}} will ensure that:
\begin{itemize}
    \item All personnel receive OHS induction training before commencing work.
    \item Task-specific training is provided for all construction activities.
    \item Specialized training is provided for high-risk work.
\end{itemize}

\section{Policy Review}
This Construction OHS Policy will be reviewed at least every two years or as necessitated by changes in legislation, company operations, or incident trends.

\section{Signature}
\textbf{Signed:}

{{CEO_NAME}}\\
\textbf{CEO, {{COMPANY_NAME}}}\\
\textbf{Date:} {{DATE}}

\section{Contact Information}
For inquiries, contact:
\begin{itemize}
    \item Phone: {{COMPANY_CONTACT}}
    \item Email: {{COMPANY_EMAIL}}
\end{itemize}

\section{Document Control}
\begin{table}[h]
    \centering
    \begin{tabular}{p{3cm}p{3cm}p{6cm}}
        \toprule
        \textbf{Version} & \textbf{Date} & \textbf{Changes} \\
        \midrule
        {{REVISION}} & {{DATE}} & Initial version \\
        \bottomrule
    \end{tabular}
    \caption{Revision History}
\end{table}

\section{Compiled By}
\begin{itemize}
    \item \textbf{Name}: {{COMPILED_BY_NAME}}
    \item \textbf{Contact}: {{COMPILED_BY_CONTACT}}
    \item \textbf{Email}: {{COMPILED_BY_EMAIL}}
    \item \textbf{Role}: {{COMPILED_BY_ROLE}}
\end{itemize}

\end{document}
