\documentclass[12pt]{article}
\usepackage[utf8]{inputenc}
\usepackage[T1]{fontenc}
\usepackage{geometry}
\geometry{a4paper, margin=1in}
\usepackage{enumitem}
\usepackage{booktabs}
\usepackage{hyperref}
\usepackage{titlesec}
\usepackage{noto}

\titleformat{\section}{\large\bfseries}{\thesection}{1em}{}
\titleformat{\subsection}{\normalsize\bfseries}{\thesubsection}{1em}{}
\titleformat{\subsubsection}{\normalsize\itshape}{\thesubsubsection}{1em}{}

\begin{document}

\begin{titlepage}
    \centering
    \vspace*{2cm}
    {\LARGE\bfseries Construction Occupational Health and Safety (OHS) Management System Standard for {{COMPANY_NAME}}\par}
    \vspace{1cm}
    {\large\itshape Standard Number: COHSMS-001\par}
    \vspace{0.5cm}
    {\normalsize Version: {{REVISION}}\par}
    \vspace{0.5cm}
    {\normalsize Effective Date: {{DATE}}\par}
    \vspace{0.5cm}
    {\normalsize Review Date: {{REVIEW_DATE}}\par}
    \vspace{2cm}
    {\normalsize Approved by: {{CEO_NAME}}, CEO\par}
\end{titlepage}

\section{Introduction}
This Construction Occupational Health and Safety (OHS) Management System Standard establishes the framework and minimum requirements for managing OHS throughout all construction projects undertaken by {{COMPANY_NAME}}. It provides a systematic approach to identifying, controlling, and minimizing OHS risks associated with construction activities, ensuring compliance with legal obligations, and promoting a culture of safety excellence.

\section{Purpose}
The purpose of this standard is to:
\begin{itemize}
    \item Define the structure and core elements of {{COMPANY_NAME}}'s Construction OHS Management System.
    \item Outline the processes for effective OHS management across all phases of construction projects.
    \item Ensure consistent application of OHS best practices and compliance with legal requirements.
    \item Provide a framework for the development of project-specific Health and Safety (H\&S) Plans.
    \item Drive continual improvement in construction OHS performance.
\end{itemize}

\section{Scope}
This standard applies to all construction projects and related activities undertaken or managed by {{COMPANY_NAME}}, including those performed by employees, contractors, and subcontractors.

\section{Definitions}
Refer to the {{COMPANY_NAME}} Construction OHS Policy for key definitions. Additional definitions specific to this standard include:
\begin{itemize}
    \item \textbf{OHS Management System:} A set of interrelated elements to establish an OHS policy and objectives, and processes to achieve those objectives.
    \item \textbf{Principal Contractor (PC):} An employer appointed by the client to perform construction work, as defined in the Construction Regulations.
    \item \textbf{Safe Work Procedure (SWP):} A documented procedure detailing how a specific task is to be carried out safely.
\end{itemize}

\section{Legal and Standard References}
This standard is based on and aims to comply with:
\begin{itemize}
    \item \textbf{South African Legislation:}
    \begin{itemize}
        \item Occupational Health and Safety Act 85 of 1993.
        \item Construction Regulations, 2014.
    \end{itemize}
    \item \textbf{South African National Standards (SANS):}
    \begin{itemize}
        \item SANS 45001: Occupational Health and Safety Management Systems.
        \item SANS 10400: National Building Regulations.
    \end{itemize}
    \item \textbf{Professional Bodies:}
    \begin{itemize}
        \item South African Council for the Project and Construction Management Professions (SACPCMP).
    \end{itemize}
\end{itemize}

\section{Construction OHS Management System Framework}

\subsection{Leadership and Worker Participation}
Top management shall demonstrate leadership and commitment to OHS in construction, ensuring roles and responsibilities are clearly defined.

\subsection{Planning for Construction OHS}

\subsubsection{Actions to Address Risks and Opportunities}
\begin{itemize}
    \item \textbf{Hazard Identification:} Systematic identification of hazards associated with all construction activities.
    \item \textbf{Risk Assessment:} Conducting baseline and issue-based risk assessments for all construction activities.
\end{itemize}

\subsubsection{Legal and Other Requirements}
A process shall be established to identify and comply with all applicable OHS legal requirements.

\subsection{Support for Construction OHS}

\subsubsection{Resources}
{{COMPANY_NAME}} shall provide necessary resources for effective OHS management on construction projects.

\subsubsection{Competence}
All personnel shall be competent, with verified qualifications for legally required appointments.

\subsection{Operation}

\subsubsection{Operational Planning and Control}
\begin{itemize}
    \item Develop and implement project-specific H\&S Plans.
    \item Ensure compliance with the Construction Regulations for high-risk activities (e.g., fall protection, excavation).
\end{itemize}

\subsection{Performance Evaluation}
OHS performance shall be monitored through regular site inspections, audits, and management reviews.

\subsection{Improvement}
Incidents and nonconformities shall be investigated, with corrective actions implemented.

\section{Health and Safety File}
A project-specific H\&S File shall be compiled and maintained for each construction project.

\section{Standard Review}
This standard will be reviewed at least every two years or as necessitated by changes in legislation or operations.

\section{Signature}
\textbf{Approved By:}

{{CEO_NAME}}\\
\textbf{CEO, {{COMPANY_NAME}}}\\
\textbf{Date:} {{DATE}}

\section{Contact Information}
For inquiries, contact:
\begin{itemize}
    \item Phone: {{COMPANY_CONTACT}}
    \item Email: {{COMPANY_EMAIL}}
\end{itemize}

\section{Document Control}
\begin{table}[h]
    \centering
    \begin{tabular}{p{3cm}p{3cm}p{6cm}}
        \toprule
        \textbf{Version} & \textbf{Date} & \textbf{Changes} \\
        \midrule
        {{REVISION}} & {{DATE}} & Initial version \\
        \bottomrule
    \end{tabular}
    \caption{Revision History}
\end{table}

\section{Compiled By}
\begin{itemize}
    \item \textbf{Name}: {{COMPILED_BY_NAME}}
    \item \textbf{Contact}: {{COMPILED_BY_CONTACT}}
    \item \textbf{Email}: {{COMPILED_BY_EMAIL}}
    \item \textbf{Role}: {{COMPILED_BY_ROLE}}
\end{itemize}

\end{document}
