\documentclass[12pt]{article}
\usepackage[utf8]{inputenc}
\usepackage[T1]{fontenc}
\usepackage{geometry}
\geometry{a4paper, margin=1in}
\usepackage{enumitem}
\usepackage{booktabs}
\usepackage{hyperref}
\usepackage{titlesec}
\usepackage{noto}

\titleformat{\section}{\large\bfseries}{\thesection}{1em}{}
\titleformat{\subsection}{\normalsize\bfseries}{\thesubsection}{1em}{}
\titleformat{\subsubsection}{\normalsize\itshape}{\thesubsubsection}{1em}{}

\begin{document}

\begin{titlepage}
    \centering
    \vspace*{2cm}
    {\LARGE\bfseries Occupational Health and Safety (OHS) Audit and Inspection Standard for {{COMPANY_NAME}}\par}
    \vspace{1cm}
    {\large\itshape Standard Number: OHSAIS-001\par}
    \vspace{0.5cm}
    {\normalsize Version: {{REVISION}}\par}
    \vspace{0.5cm}
    {\normalsize Effective Date: {{DATE}}\par}
    \vspace{0.5cm}
    {\normalsize Review Date: {{REVIEW_DATE}}\par}
    \vspace{2cm}
    {\normalsize Approved by: {{CEO_NAME}}, CEO\par}
\end{titlepage}

\section{Introduction}
This Occupational Health and Safety (OHS) Audit and Inspection Standard provides the operational framework and minimum requirements for implementing the {{COMPANY_NAME}} OHS Audit and Inspection Policy. It details the processes for planning, conducting, reporting, and following up on OHS system audits and workplace inspections to ensure a consistent, effective, and systematic approach to evaluating OHS compliance and performance.

\section{Purpose}
The purpose of this standard is to:
\begin{itemize}
    \item Define the methodologies for conducting OHS audits and inspections.
    \item Establish clear responsibilities for all stages of the audit and inspection processes.
    \item Ensure compliance with legal requirements and alignment with SANS 45001 and ISO 19011.
    \item Provide tools and guidance for conducting effective audits and inspections.
    \item Outline the process for managing nonconformities and ensuring corrective actions.
\end{itemize}

\section{Scope}
This standard applies to OHS system audits (internal and external) and workplace OHS inspections across all {{COMPANY_NAME}} facilities, sites, projects, and operations.

\section{Definitions}
Refer to the {{COMPANY_NAME}} OHS Audit and Inspection Policy for key definitions. Additional definitions specific to this standard include:
\begin{itemize}
    \item \textbf{Audit Program:} Arrangements for a set of audits planned for a specific time frame and directed towards a specific purpose.
    \item \textbf{Audit Plan:} Description of the activities and arrangements for an audit.
    \item \textbf{Inspector:} A person assigned to conduct an OHS inspection.
\end{itemize}

\section{Legal and Standard References}
This standard is guided by:
\begin{itemize}
    \item Occupational Health and Safety Act 85 of 1993.
    \item Construction Regulations, 2014.
    \item SANS 45001: Occupational Health and Safety Management Systems.
    \item ISO 19011: Guidelines for Auditing Management Systems.
\end{itemize}

\section{OHS Inspection Program}

\subsection{Purpose of Inspections}
Workplace OHS inspections are conducted to identify hazards, monitor the effectiveness of controls, and verify compliance with legal and company standards.

\subsection{Types and Frequency of Inspections}
\begin{itemize}
    \item \textbf{General Workplace Inspections:} Monthly for high-risk areas, quarterly for low-risk areas.
    \item \textbf{Specific Equipment Inspections:} Firefighting equipment (monthly), first aid boxes (weekly/monthly).
    \item \textbf{Statutory Inspections:} As required by regulations (e.g., scaffold inspections weekly).
\end{itemize}

\subsection{Planning and Scheduling Inspections}
An inspection schedule shall be developed by the OHS Department in consultation with relevant parties.

\subsection{Conducting Inspections}
\begin{itemize}
    \item \textbf{Preparation:} Review previous reports and checklists.
    \item \textbf{During the Inspection:} Use checklists, observe practices, and document findings.
    \item \textbf{Post-Inspection:} Discuss findings with supervisors and finalize reports.
\end{itemize}

\subsection{Inspection Reporting and Follow-up}
Findings shall be documented on a standardized Inspection Report Form. Corrective actions shall be tracked and verified.

\section{OHS System Audit Program}

\subsection{Purpose of OHS System Audits}
OHS system audits evaluate the effectiveness of the OHS Management System, verify compliance, and identify areas for improvement.

\subsection{Audit Program Management}
\begin{itemize}
    \item \textbf{Establishing the Audit Program:} The OHS Manager shall develop an annual audit schedule.
    \item \textbf{Audit Objectives, Scope, and Criteria:} Define objectives, scope, and criteria for each audit.
    \item \textbf{Selection of Audit Teams:} Auditors shall be competent and independent.
\end{itemize}

\subsection{Conducting an OHS System Audit}
\begin{itemize}
    \item \textbf{Initiating the Audit:} Appoint a Lead Auditor and team.
    \item \textbf{Preparing Audit Activities:} Develop an Audit Plan and checklists.
    \item \textbf{Conducting Audit Activities:} Hold opening/closing meetings, collect evidence, and generate findings.
    \item \textbf{Preparing and Distributing the Audit Report:} Document findings in a formal Audit Report.
\end{itemize}

\subsection{Audit Follow-up and Closure}
Corrective actions shall be implemented by the auditee and verified by the auditor.

\section{Development and Use of Checklists}
Checklists shall be developed based on legal requirements, standards, and risk assessments, and used as a guide during inspections and audits.

\section{Standard Review}
This standard will be reviewed at least every two years or as necessitated by changes in legislation or operations.

\section{Signature}
\textbf{Approved By:}

{{CEO_NAME}}\\
\textbf{CEO, {{COMPANY_NAME}}}\\
\textbf{Date:} {{DATE}}

\section{Contact Information}
For inquiries, contact:
\begin{itemize}
    \item Phone: {{COMPANY_CONTACT}}
    \item Email: {{COMPANY_EMAIL}}
\end{itemize}

\section{Document Control}
\begin{table}[h]
    \centering
    \begin{tabular}{p{3cm}p{3cm}p{6cm}}
        \toprule
        \textbf{Version} & \textbf{Date} & \textbf{Changes} \\
        \midrule
        {{REVISION}} & {{DATE}} & Initial version \\
        \bottomrule
    \end{tabular}
    \caption{Revision History}
\end{table}

\section{Compiled By}
\begin{itemize}
    \item \textbf{Name}: {{COMPILED_BY_NAME}}
    \item \textbf{Contact}: {{COMPILED_BY_CONTACT}}
    \item \textbf{Email}: {{COMPILED_BY_EMAIL}}
    \item \textbf{Role}: {{COMPILED_BY_ROLE}}
\end{itemize}

\end{document}
