\documentclass[11pt]{article}
\usepackage[utf8]{inputenc}
\usepackage[a4paper, margin=2.5cm]{geometry}
\usepackage{fancyhdr}
\usepackage{draftwatermark}
\usepackage{tocbibind}
\usepackage{tabularx}
\usepackage{booktabs}
\usepackage{graphicx}
\usepackage{hyperref}
\usepackage{noto}
\usepackage{ifthen}

% Branding variables
\newcommand{\brandingLevel}{light}
\newcommand{\userLogo}{}
\newcommand{\userCompany}{{{COMPANY_NAME}}}
\newcommand{\safetyHelpLogo}{/safety-plans/assets/images/logo.png}
\SetWatermarkText{SafetyHelp}
\SetWatermarkScale{2}
\SetWatermarkColor[gray]{0.9}

% Dynamic fields
\newcommand{\docTitle}{Occupational Health and Safety (OHS) Audit and Inspection Standard}
\newcommand{\refNumber}{OHSAIS-001}
\newcommand{\issueDate}{{{DATE}}}
\newcommand{\projectName}{}
\newcommand{\clientName}{{{COMPANY_NAME}}}
\newcommand{\clientContact}{{{COMPANY_CONTACT}}, {{COMPANY_EMAIL}}}
\newcommand{\compilerName}{{{COMPILED_BY_NAME}}}
\newcommand{\compilerRole}{{{COMPILED_BY_ROLE}}}
\newcommand{\compilerSACPCMP}{}
\newcommand{\compilerEmail}{{{COMPILED_BY_EMAIL}}}
\newcommand{\compilerPhone}{{{COMPILED_BY_CONTACT}}}
\newcommand{\compilerCompany}{}
\newcommand{\revision}{{{REVISION}}}
\newcommand{\reviewDate}{{{REVIEW_DATE}}}

% Header and Footer
\pagestyle{fancy}
\fancyhf{}
\ifthenelse{\equal{\brandingLevel}{heavy}}{
  \fancyhead[L]{\small \refNumber}
  \fancyhead[C]{\ifthen\isempty{\userLogo}{\textbf{\userCompany}}{\includegraphics[height=1.5cm]{\userLogo}} \quad \includegraphics[height=1cm]{\safetyHelpLogo}}
  \fancyhead[R]{\small \compilerName}
  \SetWatermarkLightness{0.7}
}{
  \fancyhead[L]{\small \refNumber}
  \fancyhead[C]{\ifthen\isempty{\userLogo}{\textbf{\userCompany}}{\includegraphics[height=1.5cm]{\userLogo}}}
  \fancyhead[R]{\small \compilerName}
}
\ifthenelse{\equal{\brandingLevel}{none}}{
  \SetWatermarkText{}
}{}
\fancyfoot[L]{\small \textit{Powered by SafetyHelp | Safety Personalized, Compliance Simplified}}
\fancyfoot[C]{\small \textit{Visit safetyfirst.help | Email: salatiso@safetyfirst.help}}
\fancyfoot[R]{\small Page \thepage\ of \pageref{LastPage}}
\renewcommand{\headrulewidth}{0.4pt}
\renewcommand{\footrulewidth}{0pt}
\fancyfoot[C]{\parbox[b]{\textwidth}{\footnotesize \textit{This document is generated by SafetyHelp for OHS compliance. Verify applicability with local regulations.}}}

\begin{document}

% Cover Page
\begin{titlepage}
  \centering
  \ifthenelse{\equal{\brandingLevel}{heavy} \OR \equal{\brandingLevel}{light}}{
    \includegraphics[height=1.5cm,keepaspectratio]{\safetyHelpLogo}\hfill
  }{}
  \vspace{2cm}
  \ifthen\isempty{\userLogo}{\textbf{\Large \userCompany}\vspace{1cm}}{\includegraphics[height=1.5cm]{\userLogo}\vspace{1cm}}
  {\Huge \textbf{\docTitle}}\vspace{1cm}
  {\Large Reference: \refNumber}\vspace{0.5cm}
  {\large Issue Date: \issueDate}\vspace{0.5cm}
  {\large Review Date: \reviewDate}\vspace{0.5cm}
  {\large Client: \clientName}\vspace{0.5cm}
  {\large \clientContact}\vspace{0.5cm}
  {\large Compiled By: \compilerName, \compilerRole}\vspace{0.5cm}
  {\large \compilerEmail, \compilerPhone}
  \vspace{2cm}
  \rule{\textwidth}{0.5pt}
\end{titlepage}

% Table of Contents
\tableofcontents
\newpage

% Introduction
\section{Introduction}
This Occupational Health and Safety (OHS) Audit and Inspection Standard provides the operational framework for implementing the \clientName\ OHS Audit and Inspection Policy. It outlines procedures for conducting audits and inspections to ensure compliance and improve OHS performance.

% Compiled By Details
\section{Compiled By Details}
\begin{tabularx}{\textwidth}{lX}
  \toprule
  \textbf{Field} & \textbf{Details} \\
  \midrule
  Name & \compilerName \\
  Role & \compilerRole \\
  SACPCMP Number & \compilerSACPCMP \\
  Email & \compilerEmail \\
  Phone & \compilerPhone \\
  Date & \issueDate \\
  Company & \compilerCompany \\
  \bottomrule
\end{tabularx}

% Revision History
\section{Revision History}
\begin{tabularx}{\textwidth}{lXll}
  \toprule
  \textbf{Rev} & \textbf{Date} & \textbf{Changes} & \textbf{Revised By} \\
  \midrule
  \revision & \issueDate & Initial version & \compilerName \\
  \bottomrule
\end{tabularx}

% Content-Specific Sections
\section{Purpose}
The purpose of this standard is to:
\begin{itemize}
    \item Define procedures for planning and conducting OHS audits and inspections.
    \item Establish responsibilities for audit and inspection activities.
    \item Ensure compliance with OHSA 85 of 1993 and SANS 45001.
    \item Provide a framework for addressing non-conformances.
\end{itemize}

\section{Scope}
This standard applies to all \clientName\ facilities, operations, and personnel in South Africa.

\section{Definitions}
Refer to the \clientName\ OHS Audit and Inspection Policy for key definitions. Additional definitions include:
\begin{itemize}
    \item \textbf{Audit Schedule:} A planned timeline for conducting audits.
    \item \textbf{Inspection Checklist:} A tool used to guide workplace inspections.
\end{itemize}

\section{Legal and Standard References}
This standard is guided by:
\begin{itemize}
    \item Occupational Health and Safety Act 85 of 1993.
    \item SANS 45001: Occupational Health and Safety Management Systems (Clause 9.2).
    \item ISO 19011: Guidelines for Auditing Management Systems.
\end{itemize]

\section{OHS Audit and Inspection Process}

\subsection{Audit Planning}
The OHS Manager shall develop an annual audit schedule, identifying audit scope, criteria, and auditors.

\subsection{Conducting Audits}
Audits shall follow ISO 19011 guidelines, including document reviews, interviews, and observations.

\subsection{Inspection Procedures}
Inspections shall be conducted monthly using standardized checklists, covering workplace conditions and compliance.

### Reporting and Follow-Up
Audit and inspection findings shall be documented, with corrective actions assigned and tracked.

\subsection{Training}
Auditors and inspectors shall be trained per the OHS Training and Competency Policy.

\section{Record Keeping}
Maintain records of audit schedules, inspection checklists, findings, and corrective actions for at least 5 years.

\section{Standard Review}
This standard will be reviewed at least every two years or as necessitated by changes in legislation or audit findings.

% Signature Page
\section{Signature Page}
\begin{itemize}
  \item \textbf{Compiler}: \compilerName \\
    Signature: \rule{5cm}{0.4pt} \quad Date: \rule{3cm}{0.4pt}
  \item \textbf{Approved By}: {{CEO_NAME}} \\
    Signature: \rule{5cm}{0.4pt} \quad Date: \rule{3cm}{0.4pt}
\end{itemize}

% Back Page
\ifthenelse{\equal{\brandingLevel}{heavy} \OR \equal{\brandingLevel}{light} \OR \equal{\brandingLevel}{none}}{
  \newpage
  \section*{About SafetyHelp}
  SafetyHelp is South Africa’s leading OHS platform, offering AI-powered tools for compliance, risk management, and safety education across industries like construction, mining, education, and healthcare.

  \textbf{Comprehensive Services}: \\
  \textit{OHS Tools}: \\
  \begin{itemize}
    \item \textit{Construction Safety}: Generate H\&S specifications, plans, risk assessments, legal appointments (OHSA, COIDA compliance).
    \item \textit{Risk Management Plans}: Step-by-step wizard with ISO 45001 integration.
    \item \textit{Document Library}: Policies, procedures, templates, permits, records (downloadable PDFs).
  \end{itemize}
  \textit{Training and Education}: \\
  \begin{itemize}
    \item \textit{Training Wizard}: Tailored programs by industry, audience, location ledge, skills, and competency to perform their tasks safely. This includes:

- **Induction Training**: All new employees and contractors receive OHS induction before commencing work, covering site-specific hazards, emergency procedures, and company OHS policies.
- **Task-Specific Training**: Training tailored to specific job roles, such as operating heavy machinery, working at heights, or handling hazardous substances, ensuring workers are equipped to perform high-risk tasks safely.
- **Hazard-Specific Training**: Focused training on identified workplace hazards, such as chemical handling, electrical safety, or confined space entry.
- **Emergency Preparedness Training**: Training on evacuation procedures, first aid, and fire response to prepare workers for emergency situations.
- **Refresher Training**: Periodic training to maintain and update knowledge and skills, ensuring ongoing competency.
- **Toolbox Talks**: Regular, short discussions on specific OHS topics to reinforce safety awareness.

### Competency Assessment
Competency is verified through assessments, which may include:
- **Practical Tests**: Demonstrations of skills in real or simulated work conditions.
- **Written or Oral Exams**: Testing knowledge of OHS procedures and regulations.
- **Observations**: Supervisor evaluations of workers performing tasks.
- **Certifications**: Obtaining recognized qualifications for specific roles (e.g., scaffolding inspector, first aider).

### Training Records
All training and competency records are maintained in the OHS Training and Competency Log, including:
- Employee details (name, employee number, job role).
- Training course details (title, date completed, certificate number, expiry date).
- Competency assessment results.

These records are kept for at least 5 years and are accessible for audits and compliance verification.

### Training Effectiveness
The effectiveness of training programs is evaluated through:
- **Feedback Surveys**: Collecting participant feedback on training content and delivery.
- **Performance Reviews**: Assessing improvements in workplace safety and task performance post-training.
- **Incident Analysis**: Monitoring whether training reduces incident rates related to specific hazards.

## Monitoring and Review
The OHS training and competency program is monitored through:
- **Regular Reviews**: The OHS Manager conducts annual reviews of training plans and outcomes, incorporating feedback, incident data, and legal updates.
- **Audits**: Internal and external audits verify compliance with training requirements and identify gaps.
- **Key Performance Indicators (KPIs)**: Metrics such as training completion rates, competency assessment pass rates, and incident rates linked to training deficiencies are tracked.

The program is updated based on:
- Changes in legislation or industry standards.
- New hazards identified through risk assessments.
- Lessons learned from incidents or near misses.
- Feedback from employees and stakeholders.

## Continual Improvement
\clientName\ is committed to continually improving its OHS training and competency program by:
- Incorporating new training methods, such as e-learning or virtual reality simulations, to enhance engagement and effectiveness.
- Expanding training offerings to cover emerging risks, such as cybersecurity threats in hybrid work environments.
- Collaborating with accredited training providers to ensure high-quality, up-to-date content.
- Engaging employees in the development of training programs to ensure relevance and buy-in.

By maintaining a robust training and competency framework, \clientName\ ensures that all personnel are equipped to contribute to a safe and healthy workplace, reducing risks and enhancing overall OHS performance.

% Signature Page
\section{Signature Page}
\begin{itemize}
  \item \textbf{Compiler}: \compilerName \\
    Signature: \rule{5cm}{0.4pt} \quad Date: \rule{3cm}{0.4pt}
  \item \textbf{Approved By}: {{CEO_NAME}} \\
    Signature: \rule{5cm}{0.4pt} \quad Date: \rule{3cm}{0.4pt}
\end{itemize}

% Back Page
\ifthenelse{\equal{\brandingLevel}{heavy} \OR \equal{\brandingLevel}{light} \OR \equal{\brandingLevel}{none}}{
  \newpage
  \section*{About SafetyHelp}
  SafetyHelp is South Africa’s leading OHS platform, offering AI-powered tools for compliance, risk management, and safety education across industries like construction, mining, education, and healthcare.

  \textbf{Comprehensive Services}: \\
  \textit{OHS Tools}: \\
  \begin{itemize}
    \item \textit{Construction Safety}: Generate H\&S specifications, plans, risk assessments, legal appointments (OHSA, COIDA compliance).
    \item \textit{Risk Management Plans}: Step-by-step wizard with ISO 45001 integration.
    \item \textit{Document Library}: Policies, procedures, templates, permits, records (downloadable PDFs).
  \end{itemize}
  \textit{Training and Education}: \\
  \begin{itemize}
    \item \textit{Training Wizard}: Tailored programs by industry, audience, location (PDF guides, online videos Q3 2025).
    \item \textit{Training Provider Directory}: Connect with accredited providers.
  \end{itemize}
  \textit{Community Safety Tools}: \\
  \begin{itemize}
    \item \textit{Incident Reporting}: Report hazards (e.g., potholes, equipment issues) with photos, location, weather data, share via email/SMS/WhatsApp/X.
    \item \textit{Safety Checklists}: For spaza shops, schools, hospitals, taverns.
  \end{itemize}
  \textit{Multilingual Accessibility}: Supports 15 languages (English, Zulu, Xhosa, Afrikaans, Mandarin, etc.) for inclusivity. \\
  \textit{User Dashboard}: Profile management, saved resources (RMPs, incident reports), supplier directory with reviews. \\
  \textit{Upcoming Features}: \\
  \begin{itemize}
    \item Online Video Training (Q3 2025)
    \item Mobile App (iOS/Android, Q4 2025)
    \item AI Chatbot Enhancements (Q2 2026)
  \end{itemize}

  \textbf{Contact}: Visit \href{https://safetyfirst.help}{safetyfirst.help} or email \href{mailto:salatiso@safetyfirst.help}{salatiso@safetyfirst.help} to learn more. Become a member to customize documents and remove branding!

  \vspace{2cm}
  \hfill \includegraphics[height=2cm]{\safetyHelpLogo}
}{}

\label{LastPage}
\end{document}
