\documentclass[12pt]{article}
\usepackage[utf8]{inputenc}
\usepackage[T1]{fontenc}
\usepackage{geometry}
\geometry{a4paper, margin=1in}
\usepackage{enumitem}
\usepackage{booktabs}
\usepackage{hyperref}
\usepackage{titlesec}
\usepackage{noto}

\titleformat{\section}{\large\bfseries}{\thesection}{1em}{}
\titleformat{\subsection}{\normalsize\bfseries}{\thesubsection}{1em}{}
\titleformat{\subsubsection}{\normalsize\itshape}{\thesubsubsection}{1em}{}

\begin{document}

\begin{titlepage}
    \centering
    \vspace*{2cm}
    {\LARGE\bfseries Occupational Health and Safety (OHS) Training and Competency Policy for {{COMPANY_NAME}}\par}
    \vspace{1cm}
    {\large\itshape Policy Number: OHSTCP-001\par}
    \vspace{0.5cm}
    {\normalsize Version: {{REVISION}}\par}
    \vspace{0.5cm}
    {\normalsize Effective Date: {{DATE}}\par}
    \vspace{0.5cm}
    {\normalsize Review Date: {{REVIEW_DATE}}\par}
    \vspace{2cm}
    {\normalsize Approved by: {{CEO_NAME}}, CEO\par}
\end{titlepage}

\section{Purpose}
The purpose of this Occupational Health and Safety (OHS) Training and Competency Policy is to affirm {{COMPANY_NAME}}'s commitment to ensuring that all employees, contractors, and other relevant persons possess the necessary knowledge, skills, and understanding to perform their work safely and without risk to their health or the health and safety of others. This policy establishes the framework for providing appropriate OHS training, developing competence, and fostering a strong safety awareness culture.

\section{Scope}
This policy applies to all employees of {{COMPANY_NAME}} at all levels, as well as to contractors, temporary workers, and visitors where their activities or presence may impact OHS. It covers all OHS-related training, including induction, task-specific training, hazard-specific training, emergency preparedness training, refresher training, and awareness programs across all company operations and sites in South Africa.

\section{Definitions}
\begin{itemize}
    \item \textbf{Competence:} The ability to apply knowledge and skills to achieve intended results in a safe and healthy manner.
    \item \textbf{Training:} The process of teaching or learning a skill or behavior related to safe work practices and hazard control.
    \item \textbf{OHS Induction:} Initial OHS training for new employees, contractors, or visitors to familiarize them with OHS policies and procedures.
    \item \textbf{Task-Specific Training:} Training focused on the safe performance of particular jobs or tasks.
    \item \textbf{Toolbox Talk:} Short, informal OHS discussions focusing on specific hazards or safe work practices.
\end{itemize}

\section{Policy Statement}
{{COMPANY_NAME}} is committed to:
\begin{itemize}
    \item \textbf{Ensuring Competence:} Ensuring all individuals are competent to perform tasks safely.
    \item \textbf{Providing Training:} Delivering comprehensive OHS training appropriate to roles and risks.
    \item \textbf{Identifying Training Needs:} Systematically identifying training needs through risk assessments and job analyses.
    \item \textbf{Legal Compliance:} Complying with South African OHS legal requirements for training and supervision.
    \item \textbf{Induction:} Providing OHS induction to all new personnel before work commences.
    \item \textbf{Task and Hazard Specificity:} Offering specific training for high-risk tasks and hazards.
    \item \textbf{Emergency Preparedness:} Training personnel in emergency procedures.
    \item \textbf{Refresher Training:} Conducting refresher training to maintain knowledge and skills.
    \item \textbf{Record Keeping:} Maintaining accurate training and competency records.
    \item \textbf{Evaluation of Effectiveness:} Evaluating training programs for effectiveness.
    \item \textbf{Resource Allocation:} Providing adequate resources for OHS training.
    \item \textbf{Continual Improvement:} Reviewing and improving training programs.
    \item \textbf{Promoting Awareness:} Fostering OHS awareness through communication.
\end{itemize}

\section{Roles and Responsibilities}
\begin{itemize}
    \item \textbf{Top Management (CEO, {{CEO_NAME}}):}
    \begin{itemize}
        \item Overall accountability for the training policy implementation.
        \item Allocating resources for training.
        \item Promoting a culture valuing OHS training.
    \end{itemize}
    \item \textbf{OHS Manager ({{COMPILED_BY_NAME}}):}
    \begin{itemize}
        \item Developing and maintaining the training framework.
        \item Overseeing Training Needs Analysis and delivery.
        \item Maintaining training records.
    \end{itemize}
    \item \textbf{Line Managers and Supervisors:}
    \begin{itemize}
        \item Identifying training needs for their teams.
        \item Ensuring team members receive required training.
        \item Verifying competence before task assignment.
    \end{itemize}
    \item \textbf{Employees and Contractors:}
    \begin{itemize}
        \item Participating in required OHS training.
        \item Applying training knowledge in daily work.
        \item Reporting training gaps.
    \end{itemize}
\end{itemize}

\section{Key Principles for OHS Training and Competency}
\begin{itemize}
    \item \textbf{Relevance:} Training content will be relevant to tasks and risks.
    \item \textbf{Practicality:} Training will include practical elements.
    \item \textbf{Understandability:} Training will suit participants’ literacy and language.
    \item \textbf{Verification:} Competence will be verified appropriately.
    \item \textbf{Accessibility:} Training will be accessible to all relevant personnel.
\end{itemize}

\section{Legal and Standard Framework}
This policy is guided by:
\begin{itemize}
    \item Occupational Health and Safety Act 85 of 1993 (Sections 8, 13).
    \item Construction Regulations, 2014.
    \item General Safety Regulations.
    \item Basic Conditions of Employment Act 75 of 1997.
    \item SANS 45001: Occupational Health and Safety Management Systems (Clauses 7.2, 7.3).
\end{itemize}

\section{Policy Review}
This OHS Training and Competency Policy will be reviewed at least every two years or as necessitated by changes in legislation, operations, or training evaluations.

\section{Signature}
\textbf{Signed:}

{{CEO_NAME}}\\
\textbf{CEO, {{COMPANY_NAME}}}\\
\textbf{Date:} {{DATE}}

\section{Contact Information}
For inquiries, contact:
\begin{itemize}
    \item Phone: {{COMPANY_CONTACT}}
    \item Email: {{COMPANY_EMAIL}}
\end{itemize}

\section{Document Control}
\begin{table}[h]
    \centering
    \begin{tabular}{p{3cm}p{3cm}p{6cm}}
        \toprule
        \textbf{Version} & \textbf{Date} & \textbf{Changes} \\
        \midrule
        {{REVISION}} & {{DATE}} & Initial version \\
        \bottomrule
    \end{tabular}
    \caption{Revision History}
\end{table}

\section{Compiled By}
\begin{itemize}
    \item \textbf{Name}: {{COMPILED_BY_NAME}}
    \item \textbf{Contact}: {{COMPILED_BY_CONTACT}}
    \item \textbf{Email}: {{COMPILED_BY_EMAIL}}
    \item \textbf{Role}: {{COMPILED_BY_ROLE}}
\end{itemize}

\end{document}
