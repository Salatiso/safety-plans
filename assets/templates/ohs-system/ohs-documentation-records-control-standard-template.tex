\documentclass[12pt]{article}
\usepackage[utf8]{inputenc}
\usepackage[T1]{fontenc}
\usepackage{geometry}
\geometry{a4paper, margin=1in}
\usepackage{enumitem}
\usepackage{booktabs}
\usepackage{hyperref}
\usepackage{titlesec}
\usepackage{noto}

\titleformat{\section}{\large\bfseries}{\thesection}{1em}{}
\titleformat{\subsection}{\normalsize\bfseries}{\thesubsection}{1em}{}
\titleformat{\subsubsection}{\normalsize\itshape}{\thesubsubsection}{1em}{}

\begin{document}

\begin{titlepage}
    \centering
    \vspace*{2cm}
    {\LARGE\bfseries Occupational Health and Safety (OHS) Documentation and Records Control Standard for {{COMPANY_NAME}}\par}
    \vspace{1cm}
    {\large\itshape Standard Number: OHSDRCS-001\par}
    \vspace{0.5cm}
    {\normalsize Version: {{REVISION}}\par}
    \vspace{0.5cm}
    {\normalsize Effective Date: {{DATE}}\par}
    \vspace{0.5cm}
    {\normalsize Review Date: {{REVIEW_DATE}}\par}
    \vspace{2cm}
    {\normalsize Approved by: {{CEO_NAME}}, CEO\par}
\end{titlepage}

\section{Introduction}
This Occupational Health and Safety (OHS) Documentation and Records Control Standard provides the operational framework for implementing the {{COMPANY_NAME}} OHS Documentation and Records Control Policy. It details processes for creating, controlling, retaining, and disposing of OHS documentation and records to ensure compliance and support the OHS Management System.

\section{Purpose}
The purpose of this standard is to:
\begin{itemize}
    \item Define procedures for document control and records management.
    \item Ensure OHS documentation is accurate, current, and accessible.
    \item Establish retention periods and secure disposal methods for OHS records.
    \item Support compliance with legal and SANS 45001 requirements.
    \item Facilitate audits and continual improvement through proper documentation.
\end{itemize}

\section{Scope}
This standard applies to all OHS documentation and records across {{COMPANY_NAME}}’s operations in South Africa.

\section{Definitions}
Refer to the {{COMPANY_NAME}} OHS Documentation and Records Control Policy for key definitions. Additional definitions include:
\begin{itemize}
    \item \textbf{Obsolete Document:} A document that is no longer valid or approved for use.
    \item \textbf{Retention Period:} The duration for which a record must be kept before disposal.
\end{itemize}

\section{Legal and Standard References}
This standard is guided by:
\begin{itemize}
    \item Occupational Health and Safety Act 85 of 1993 (Section 7.5).
    \item General Administrative Regulations (GAR 9).
    \item SANS 45001: Occupational Health and Safety Management Systems (Clause 7.5).
\end{itemize}

\section{OHS Documentation and Records Control Process}

\subsection{Document Control}
\begin{itemize}
    \item \textbf{Creation:** OHS documents (e.g., policies, procedures) shall be created by designated personnel and reviewed for accuracy.
    \item \textbf{Approval:** Documents shall be approved by the OHS Manager or Top Management before distribution.
    \item \textbf{Distribution:** Approved documents shall be distributed via a controlled system (e.g., intranet, hard copies) to relevant personnel.
    \item \textbf{Version Control:** Use a Document Control Log to track versions, ensuring only current documents are used. Obsolete documents shall be archived.
    \item \textbf{Review:** Documents shall be reviewed at least annually or after significant changes (e.g., legal updates, incidents).
\end{itemize}

\subsection{Records Management}
\begin{itemize}
    \item \textbf{Collection:** OHS records (e.g., incident reports, training logs) shall be collected and stored systematically.
    \item \textbf{Storage:** Records shall be stored securely (e.g., locked cabinets, password-protected systems) to prevent unauthorized access.
    \item \textbf{Retention:** Define retention periods per a Records Retention Schedule (e.g., incident records: 5 years; medical surveillance: 40 years).
    \item \textbf{Disposal:** Dispose of records securely (e.g., shredding, secure deletion) after the retention period, documenting the disposal process.
\end{itemize}

\subsection{Accessibility}
\begin{itemize}
    \item Ensure OHS documentation and records are accessible to authorized personnel for audits, inspections, and reviews.
    \item Protect sensitive records (e.g., medical surveillance) with restricted access.
\end{itemize}

\subsection{Training}
Provide annual training to employees on document control and records management procedures.

\section{Record Keeping}
Maintain a Document Control Log and Records Retention Schedule as evidence of compliance with this standard.

\section{Standard Review}
This standard will be reviewed at least every two years or as necessitated by changes in legislation, operations, or system requirements.

\section{Signature}
\textbf{Approved By:}

{{CEO_NAME}}\\
\textbf{CEO, {{COMPANY_NAME}}}\\
\textbf{Date:} {{DATE}}

\section{Contact Information}
For inquiries, contact:
\begin{itemize}
    \item Phone: {{COMPANY_CONTACT}}
    \item Email: {{COMPANY_EMAIL}}
\end{itemize}

\section{Document Control}
\begin{table}[h]
    \centering
    \begin{tabular}{p{3cm}p{3cm}p{6cm}}
        \toprule
        \textbf{Version} & \textbf{Date} & \textbf{Changes} \\
        \midrule
        {{REVISION}} & {{DATE}} & Initial version \\
        \bottomrule
    \end{tabular}
    \caption{Revision History}
\end{table}

\section{Compiled By}
\begin{itemize}
    \item \textbf{Name}: {{COMPILED_BY_NAME}}
    \item \textbf{Contact}: {{COMPILED_BY_CONTACT}}
    \item \textbf{Email}: {{COMPILED_BY_EMAIL}}
    \item \textbf{Role}: {{COMPILED_BY_ROLE}}
\end{itemize}

\end{document}
