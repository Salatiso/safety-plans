\documentclass[12pt]{article}
\usepackage[utf8]{inputenc}
\usepackage[T1]{fontenc}
\usepackage{geometry}
\geometry{a4paper, margin=1in}
\usepackage{enumitem}
\usepackage{booktabs}
\usepackage{hyperref}
\usepackage{titlesec}
\usepackage{noto}

\titleformat{\section}{\large\bfseries}{\thesection}{1em}{}
\titleformat{\subsection}{\normalsize\bfseries}{\thesubsection}{1em}{}
\titleformat{\subsubsection}{\normalsize\itshape}{\thesubsubsection}{1em}{}

\begin{document}

\begin{titlepage}
    \centering
    \vspace*{2cm}
    {\LARGE\bfseries Occupational Health and Safety (OHS) Audit and Inspection Policy for {{COMPANY_NAME}}\par}
    \vspace{1cm}
    {\large\itshape Policy Number: OHSAI-001\par}
    \vspace{0.5cm}
    {\normalsize Version: {{REVISION}}\par}
    \vspace{0.5cm}
    {\normalsize Effective Date: {{DATE}}\par}
    \vspace{0.5cm}
    {\normalsize Review Date: {{REVIEW_DATE}}\par}
    \vspace{2cm}
    {\normalsize Approved by: {{CEO_NAME}}, CEO\par}
\end{titlepage}

\section{Purpose}
The purpose of this Occupational Health and Safety (OHS) Audit and Inspection Policy is to affirm {{COMPANY_NAME}}'s commitment to proactively monitoring, evaluating, and improving OHS performance through a systematic program of OHS audits and inspections. This policy establishes the framework for verifying compliance with legal requirements, company OHS standards, and the OHS Management System, as well as for identifying hazards, nonconformities, and opportunities for improvement.

\section{Scope}
This policy applies to all workplaces, activities, processes, and OHS management system elements within {{COMPANY_NAME}} in South Africa. It covers internal OHS inspections, internal OHS audits, and external OHS audits conducted by regulatory bodies or third-party auditors.

\section{Definitions}
\begin{itemize}
    \item \textbf{Audit:} A systematic, independent, and documented process for obtaining audit evidence and evaluating it objectively to determine the extent to which audit criteria are fulfilled.
    \item \textbf{Inspection:} A systematic examination of a workplace, area, equipment, or process to identify hazards, unsafe conditions, unsafe acts, and non-compliance with established standards or legal requirements.
    \item \textbf{Nonconformity:} Non-fulfillment of a requirement (legal, standard, procedural).
    \item \textbf{Corrective Action:} Action to eliminate the cause of a detected nonconformity or other undesirable situation.
\end{itemize}

\section{Policy Statement}
{{COMPANY_NAME}} is committed to:
\begin{itemize}
    \item \textbf{Systematic Evaluation:} Establishing a comprehensive program of OHS audits and inspections to evaluate OHS performance and compliance.
    \item \textbf{Compliance Verification:} Verifying compliance with South African OHS legislation, SANS standards, and company OHS policies through audits and inspections.
    \item \textbf{Hazard Identification and Risk Control:} Using inspections to proactively identify workplace hazards and ensure risks are controlled.
    \item \textbf{Management System Effectiveness:} Conducting OHS audits to assess the effectiveness of the OHS Management System and facilitate continual improvement.
    \item \textbf{Objectivity and Independence:} Ensuring audits are conducted by competent personnel independent of the activity being audited.
    \item \textbf{Competence:} Ensuring personnel conducting audits and inspections are competent.
    \item \textbf{Resource Allocation:} Providing adequate resources for planning, execution, and follow-up of audits and inspections.
    \item \textbf{Corrective and Preventive Actions:} Addressing findings from audits and inspections through timely corrective actions.
    \item \textbf{Continual Improvement:} Using audit and inspection results to drive continual improvement of the OHS Management System.
    \item \textbf{Consultation and Participation:} Involving employees and Health and Safety Representatives in inspections and communicating outcomes.
\end{itemize}

\section{Roles and Responsibilities}
\begin{itemize}
    \item \textbf{Top Management (CEO, {{CEO_NAME}}):}
    \begin{itemize}
        \item Overall accountability for the OHS audit and inspection program.
        \item Ensuring resources are allocated.
        \item Reviewing significant audit findings.
    \end{itemize}
    \item \textbf{OHS Manager ({{COMPILED_BY_NAME}}):}
    \begin{itemize}
        \item Developing, implementing, and managing the OHS audit and inspection program.
        \item Ensuring auditors and inspectors are competent.
        \item Tracking corrective actions.
    \end{itemize}
    \item \textbf{Line Managers and Supervisors:}
    \begin{itemize}
        \item Facilitating audits and inspections in their areas.
        \item Implementing corrective actions.
        \item Conducting regular workplace inspections.
    \end{itemize}
    \item \textbf{Employees:}
    \begin{itemize}
        \item Cooperating with auditors and inspectors.
        \item Reporting hazards and OHS concerns.
    \end{itemize}
\end{itemize}

\section{Key Principles for OHS Audits and Inspections}
\begin{itemize}
    \item \textbf{Planned and Systematic:} Audits and inspections will be planned and conducted systematically.
    \item \textbf{Risk-Based Approach:} Frequency and scope will consider the level of OHS risk.
    \item \textbf{Evidence-Based:} Findings will be based on objective evidence.
    \item \textbf{Focus on Improvement:} Aim to identify areas for improvement.
    \item \textbf{Follow-up:} Corrective actions will be tracked for effective implementation.
\end{itemize}

\section{Legal and Standard Framework}
This policy is guided by:
\begin{itemize}
    \item Occupational Health and Safety Act 85 of 1993.
    \item Construction Regulations, 2014.
    \item SANS 45001: Occupational Health and Safety Management Systems.
    \item ISO 19011: Guidelines for Auditing Management Systems.
\end{itemize}

\section{Policy Review}
This OHS Audit and Inspection Policy will be reviewed at least every two years or as necessitated by changes in legislation, operations, or lessons learned.

\section{Signature}
\textbf{Signed:}

{{CEO_NAME}}\\
\textbf{CEO, {{COMPANY_NAME}}}\\
\textbf{Date:} {{DATE}}

\section{Contact Information}
For inquiries, contact:
\begin{itemize}
    \item Phone: {{COMPANY_CONTACT}}
    \item Email: {{COMPANY_EMAIL}}
\end{itemize}

\section{Document Control}
\begin{table}[h]
    \centering
    \begin{tabular}{p{3cm}p{3cm}p{6cm}}
        \toprule
        \textbf{Version} & \textbf{Date} & \textbf{Changes} \\
        \midrule
        {{REVISION}} & {{DATE}} & Initial version \\
        \bottomrule
    \end{tabular}
    \caption{Revision History}
\end{table}

\section{Compiled By}
\begin{itemize}
    \item \textbf{Name}: {{COMPILED_BY_NAME}}
    \item \textbf{Contact}: {{COMPILED_BY_CONTACT}}
    \item \textbf{Email}: {{COMPILED_BY_EMAIL}}
    \item \textbf{Role}: {{COMPILED_BY_ROLE}}
\end{itemize}

\end{document}
