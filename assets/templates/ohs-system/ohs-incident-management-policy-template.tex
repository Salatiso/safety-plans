\documentclass[12pt]{article}
\usepackage[utf8]{inputenc}
\usepackage[T1]{fontenc}
\usepackage{geometry}
\geometry{a4paper, margin=1in}
\usepackage{enumitem}
\usepackage{booktabs}
\usepackage{hyperref}
\usepackage{titlesec}
\usepackage{noto}

\titleformat{\section}{\large\bfseries}{\thesection}{1em}{}
\titleformat{\subsection}{\normalsize\bfseries}{\thesubsection}{1em}{}
\titleformat{\subsubsection}{\normalsize\itshape}{\thesubsubsection}{1em}{}

\begin{document}

\begin{titlepage}
    \centering
    \vspace*{2cm}
    {\LARGE\bfseries Occupational Health and Safety (OHS) Incident Management Policy for {{COMPANY_NAME}}\par}
    \vspace{1cm}
    {\large\itshape Policy Number: OHSIMP-001\par}
    \vspace{0.5cm}
    {\normalsize Version: {{REVISION}}\par}
    \vspace{0.5cm}
    {\normalsize Effective Date: {{DATE}}\par}
    \vspace{0.5cm}
    {\normalsize Review Date: {{REVIEW_DATE}}\par}
    \vspace{2cm}
    {\normalsize Approved by: {{CEO_NAME}}, CEO\par}
\end{titlepage}

\section{Purpose}
The purpose of this Occupational Health and Safety (OHS) Incident Management Policy is to affirm {{COMPANY_NAME}}'s commitment to a systematic and proactive approach to the reporting, investigation, analysis, and prevention of all work-related OHS incidents. This policy aims to ensure that all incidents are managed effectively to prevent recurrence, minimize harm, comply with legal obligations, and drive continual improvement in OHS performance.

\section{Scope}
This policy applies to all OHS incidents occurring at {{COMPANY_NAME}} workplaces, during work-related activities, or affecting {{COMPANY_NAME}} employees, contractors, visitors, or members of the public as a result of our operations in South Africa, including fatalities, injuries, occupational diseases, near misses, dangerous occurrences, property damage, and environmental incidents with OHS implications.

\section{Definitions}
\begin{itemize}
    \item \textbf{OHS Incident:} An undesired event arising out of or in the course of work that results in or could have resulted in injury, ill health, fatality, damage, or other loss.
    \item \textbf{Accident:} An OHS incident that has resulted in injury, ill health, or fatality.
    \item \textbf{Near Miss:} An OHS incident where no injury, ill health, or fatality occurs, but had the potential to do so.
    \item \textbf{Lost Time Injury (LTI):} A work-related injury that results in the injured person being unable to perform their normal duties for one full shift or more after the day of the injury.
    \item \textbf{Section 24 Incident:} An incident as defined in Section 24 of the OHS Act 85 of 1993, requiring reporting to the Department of Employment and Labour.
\end{itemize}

\section{Policy Statement}
{{COMPANY_NAME}} is committed to:
\begin{itemize}
    \item \textbf{Prompt Reporting:} Ensuring all OHS incidents, including near misses, are reported promptly.
    \item \textbf{Thorough Investigation:} Conducting objective investigations to identify contributing factors and root causes.
    \item \textbf{Learning and Prevention:} Implementing corrective and preventive actions to prevent recurrence.
    \item \textbf{Legal Compliance:} Complying with South African legal requirements for incident reporting and investigation.
    \item \textbf{Support for Affected Persons:} Providing support to those affected by OHS incidents.
    \item \textbf{Resource Allocation:} Providing resources for effective incident management.
    \item \textbf{Competence and Training:} Ensuring personnel involved in incident management are competent.
    \item \textbf{Communication:} Sharing lessons learned to enhance awareness and prevention.
    \item \textbf{No-Blame Culture:} Fostering a culture where employees feel safe to report incidents.
    \item \textbf{Continual Improvement:} Utilizing incident data for OHS Management System improvement.
\end{itemize}

\section{Roles and Responsibilities}
\begin{itemize}
    \item \textbf{Top Management (CEO, {{CEO_NAME}}):}
    \begin{itemize}
        \item Overall accountability for the incident management system.
        \item Ensuring a culture that supports reporting and learning.
        \item Reviewing significant investigation findings.
    \end{itemize}
    \item \textbf{OHS Manager ({{COMPILED_BY_NAME}}):}
    \begin{itemize}
        \item Developing and maintaining the incident management procedure.
        \item Overseeing investigations and statutory reporting.
        \item Analyzing incident trends and reporting to management.
    \end{itemize}
    \item \textbf{Line Managers and Supervisors:}
    \begin{itemize}
        \item Ensuring prompt incident reporting in their areas.
        \item Leading or participating in investigations.
        \item Implementing corrective actions.
    \end{itemize}
    \item \textbf{All Employees and Contractors:}
    \begin{itemize}
        \item Reporting all OHS incidents immediately.
        \item Cooperating with investigations.
    \end{itemize}
\end{itemize}

\section{Key Principles for OHS Incident Management}
\begin{itemize}
    \item \textbf{Timeliness:} Incidents are reported and investigated promptly.
    \item \textbf{Objectivity:} Investigations focus on fact-finding, not fault-finding.
    \item \textbf{Thoroughness:} All contributing factors and root causes are identified.
    \item \textbf{Systemic Focus:} Investigations consider management system failures.
    \item \textbf{Learning Culture:} Incidents are opportunities to learn and improve.
    \item \textbf{Prevention-Oriented:} The goal is to prevent recurrence.
\end{itemize}

\section{Legal and Standard Framework}
This policy is guided by:
\begin{itemize}
    \item Occupational Health and Safety Act 85 of 1993 (Sections 24, 25).
    \item General Administrative Regulations (GAR 8, 9).
    \item Compensation for Occupational Injuries and Diseases Act (COIDA) 130 of 1993.
    \item SANS 45001: Occupational Health and Safety Management Systems (Clause 10.2).
\end{itemize}

\section{Policy Review}
This OHS Incident Management Policy will be reviewed at least every two years or as necessitated by changes in legislation, operations, or incident trends.

\section{Signature}
\textbf{Signed:}

{{CEO_NAME}}\\
\textbf{CEO, {{COMPANY_NAME}}}\\
\textbf{Date:} {{DATE}}

\section{Contact Information}
For inquiries, contact:
\begin{itemize}
    \item Phone: {{COMPANY_CONTACT}}
    \item Email: {{COMPANY_EMAIL}}
\end{itemize}

\section{Document Control}
\begin{table}[h]
    \centering
    \begin{tabular}{p{3cm}p{3cm}p{6cm}}
        \toprule
        \textbf{Version} & \textbf{Date} & \textbf{Changes} \\
        \midrule
        {{REVISION}} & {{DATE}} & Initial version \\
        \bottomrule
    \end{tabular}
    \caption{Revision History}
\end{table}

\section{Compiled By}
\begin{itemize}
    \item \textbf{Name}: {{COMPILED_BY_NAME}}
    \item \textbf{Contact}: {{COMPILED_BY_CONTACT}}
    \item \textbf{Email}: {{COMPILED_BY_EMAIL}}
    \item \textbf{Role}: {{COMPILED_BY_ROLE}}
\end{itemize}

\end{document}
