\documentclass[12pt]{article}
\usepackage[utf8]{inputenc}
\usepackage[T1]{fontenc}
\usepackage{geometry}
\geometry{a4paper, margin=1in}
\usepackage{enumitem}
\usepackage{booktabs}
\usepackage{hyperref}
\usepackage{titlesec}
\usepackage{noto}

\titleformat{\section}{\large\bfseries}{\thesection}{1em}{}
\titleformat{\subsection}{\normalsize\bfseries}{\thesubsection}{1em}{}
\titleformat{\subsubsection}{\normalsize\itshape}{\thesubsubsection}{1em}{}

\begin{document}

\begin{titlepage}
    \centering
    \vspace*{2cm}
    {\LARGE\bfseries Occupational Health and Safety (OHS) Communication, Participation, and Consultation Standard for {{COMPANY_NAME}}\par}
    \vspace{1cm}
    {\large\itshape Standard Number: OHSCPCS-001\par}
    \vspace{0.5cm}
    {\normalsize Version: {{REVISION}}\par}
    \vspace{0.5cm}
    {\normalsize Effective Date: {{DATE}}\par}
    \vspace{0.5cm}
    {\normalsize Review Date: {{REVIEW_DATE}}\par}
    \vspace{2cm}
    {\normalsize Approved by: {{CEO_NAME}}, CEO\par}
\end{titlepage}

\section{Introduction}
This Occupational Health and Safety (OHS) Communication, Participation, and Consultation Standard provides the operational framework for implementing the {{COMPANY_NAME}} OHS Communication, Participation, and Consultation Policy. It details processes for effective OHS communication, employee participation, stakeholder consultation, legal appointments, incident alerts, and media handling.

\section{Purpose}
The purpose of this standard is to:
\begin{itemize}
    \item Define processes for OHS communication, participation, and consultation.
    \item Establish procedures for legal appointments (e.g., OHS reps, first aiders, fire wardens).
    \item Ensure compliance with legal and SANS 45001 requirements.
    \item Facilitate incident alerts and media handling for OHS incidents.
    \item Promote a culture of safety through active involvement.
\end{itemize}

\section{Scope}
This standard applies to all OHS communication, participation, and consultation activities across {{COMPANY_NAME}}’s operations in South Africa.

\section{Definitions}
Refer to the {{COMPANY_NAME}} OHS Communication, Participation, and Consultation Policy for key definitions. Additional definitions include:
\begin{itemize}
    \item \textbf{Legal Appointment:} A formal designation under the OHS Act (e.g., OHS rep, first aider).
    \item \textbf{OHS Committee:} A group established per OHS Act Section 19 to address OHS matters.
\end{itemize}

\section{Legal and Standard References}
This standard is guided by:
\begin{itemize}
    \item Occupational Health and Safety Act 85 of 1993 (Sections 16, 17, 18, 19, 20).
    \item General Safety Regulations (GSR 3, GSR 5).
    \item SANS 45001: Occupational Health and Safety Management Systems (Clauses 5.4, 7.4).
\end{itemize}

\section{OHS Communication, Participation, and Consultation Process}

\subsection{OHS Communication}
\begin{itemize}
    \item \textbf{Internal Communication:} Share OHS information via meetings, toolbox talks, and notices.
    \item \textbf{Incident Alerts:} Issue alerts post-incident to inform and educate, using a standardized template.
    \item \textbf{External Media:} All media interactions related to OHS incidents shall be handled by the communications manager (16.1/16.2 equivalent).
\end{itemize}

\subsection{Participation}
Employees shall participate in:
\begin{itemize}
    \item Risk assessments and incident investigations.
    \item OHS committees and safety initiatives.
    \item Nominating OHS representatives.
\end{itemize}

\subsection{Consultation}
Consultation shall occur on OHS matters, including:
\begin{itemize}
    \item Changes to work processes or hazards.
    \item Development of OHS policies and procedures.
    \item Appointment of OHS representatives and other roles.
\end{itemize}

\subsection{Legal Appointments and Nomination Processes}
\begin{itemize}
    \item \textbf{OHS Representatives (OHS Act Section 17):}
        \begin{itemize}
            \item \textbf{Nomination:} Employees nominate representatives in areas with 20+ workers, ensuring representation across shifts and departments.
            \item \textbf{Appointment:} The OHS Manager facilitates elections, and management formally appoints elected reps via a written letter (Section 17(2)).
            \item \textbf{Training:} Provide training on OHS Act, hazard identification, and committee roles.
        \end{itemize}
    \item \textbf{First Aiders (GSR 3):}
        \begin{itemize}
            \item \textbf{Nomination:} Supervisors identify candidates based on willingness and availability.
            \item \textbf{Appointment:} Appoint at least one first aider per 50 employees (or per shift in high-risk areas), issuing a formal appointment letter.
            \item \textbf{Training:} Ensure certification through an accredited provider, valid for 3 years.
        \end{itemize}
    \item \textbf{Fire Wardens:}
        \begin{itemize}
            \item \textbf{Nomination:} Supervisors nominate candidates based on leadership and emergency response capability.
            \item \textbf{Appointment:} Appoint fire wardens per site/area, issuing a formal appointment letter.
            \item \textbf{Training:} Provide training on fire evacuation, extinguisher use, and emergency coordination.
        \end{itemize}
    \item \textbf{Other Roles (e.g., Incident Investigators):} Follow a similar nomination, appointment, and training process.
\end{itemize}

\subsection{OHS Committees}
\begin{itemize}
    \item \textbf{Establishment (Section 19):} Form committees in workplaces with 2+ OHS reps, including management and employee representatives.
    \item \textbf{Representation:} Ensure reps reflect all shifts, departments, and risk levels.
    \item \textbf{Meetings:} Hold quarterly meetings to discuss OHS issues, incidents, and improvements, documenting outcomes in minutes.
\end{itemize}

\section{Record Keeping}
Maintain records of appointments, committee minutes, incident alerts, and consultation activities for at least 5 years.

\section{Standard Review}
This standard will be reviewed at least every two years or as necessitated by changes in legislation, operations, or feedback.

\section{Signature}
\textbf{Approved By:}

{{CEO_NAME}}\\
\textbf{CEO, {{COMPANY_NAME}}}\\
\textbf{Date:} {{DATE}}

\section{Contact Information}
For inquiries, contact:
\begin{itemize}
    \item Phone: {{COMPANY_CONTACT}}
    \item Email: {{COMPANY_EMAIL}}
\end{itemize}

\section{Document Control}
\begin{table}[h]
    \centering
    \begin{tabular}{p{3cm}p{3cm}p{6cm}}
        \toprule
        \textbf{Version} & \textbf{Date} & \textbf{Changes} \\
        \midrule
        {{REVISION}} & {{DATE}} & Initial version \\
        \bottomrule
    \end{tabular}
    \caption{Revision History}
\end{table}

\section{Compiled By}
\begin{itemize}
    \item \textbf{Name}: {{COMPILED_BY_NAME}}
    \item \textbf{Contact}: {{COMPILED_BY_CONTACT}}
    \item \textbf{Email}: {{COMPILED_BY_EMAIL}}
    \item \textbf{Role}: {{COMPILED_BY_ROLE}}
\end{itemize}

\end{document}
