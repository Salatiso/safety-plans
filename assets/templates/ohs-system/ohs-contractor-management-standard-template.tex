\documentclass[12pt]{article}
\usepackage[utf8]{inputenc}
\usepackage[T1]{fontenc}
\usepackage{geometry}
\geometry{a4paper, margin=1in}
\usepackage{enumitem}
\usepackage{booktabs}
\usepackage{hyperref}
\usepackage{titlesec}
\usepackage{noto}

\titleformat{\section}{\large\bfseries}{\thesection}{1em}{}
\titleformat{\subsection}{\normalsize\bfseries}{\thesubsection}{1em}{}
\titleformat{\subsubsection}{\normalsize\itshape}{\thesubsubsection}{1em}{}

\begin{document}

\begin{titlepage}
    \centering
    \vspace*{2cm}
    {\LARGE\bfseries Occupational Health and Safety (OHS) Contractor Management Standard for {{COMPANY_NAME}}\par}
    \vspace{1cm}
    {\large\itshape Standard Number: OHSCMS-001\par}
    \vspace{0.5cm}
    {\normalsize Version: {{REVISION}}\par}
    \vspace{0.5cm}
    {\normalsize Effective Date: {{DATE}}\par}
    \vspace{0.5cm}
    {\normalsize Review Date: {{REVIEW_DATE}}\par}
    \vspace{2cm}
    {\normalsize Approved by: {{CEO_NAME}}, CEO\par}
\end{titlepage}

\section{Introduction}
This Occupational Health and Safety (OHS) Contractor Management Standard provides the operational framework and minimum requirements for implementing the {{COMPANY_NAME}} OHS Contractor Management Policy. It details systematic processes for selecting, managing, and monitoring contractors to ensure OHS compliance and performance throughout their engagement.

\section{Purpose}
The purpose of this standard is to:
\begin{itemize}
    \item Define procedures for contractor pre-qualification, selection, and engagement.
    \item Establish requirements for monitoring and evaluating contractor OHS performance.
    \item Ensure contractors’ OHS systems, incident records, and safety measures are assessed.
    \item Integrate OHS penalties into main project contracts per JBCC, NEC, FIDIC, and GCC.
    \item Support compliance with legal and contractual OHS requirements.
    \item Promote continual improvement in contractor OHS performance.
\end{itemize}

\section{Scope}
This standard applies to all contractors, subcontractors, and service providers engaged by {{COMPANY_NAME}} for work at company sites, projects, or facilities in South Africa.

\section{Definitions}
Refer to the {{COMPANY_NAME}} OHS Contractor Management Policy for key definitions. Additional definitions include:
\begin{itemize}
    \item \textbf{Pre-Qualification Assessment:} An evaluation of a contractor’s OHS capabilities before engagement.
    \item \textbf{OHS Performance Metrics:} Indicators such as incident rates, audit scores, and compliance levels used to evaluate contractors.
\end{itemize}

\section{Legal and Standard References}
This standard is guided by:
\begin{itemize}
    \item Occupational Health and Safety Act 85 of 1993 (Sections 8, 9, 37).
    \item Construction Regulations, 2014 (CR 5, CR 7).
    \item SANS 45001: Occupational Health and Safety Management Systems (Clause 8.1.6).
    \item JBCC Clause 12, NEC Clause 27, FIDIC Clause 4.8, GCC Clause 4.9.
\end{itemize}

\section{OHS Contractor Management Process}

\subsection{Pre-Qualification and Selection}
The OHS Manager shall conduct a pre-qualification assessment using a Contractor Evaluation Checklist, assessing:
\begin{itemize}
    \item Contractor’s OHS management system (e.g., policies, procedures, risk assessments).
    \item Historical OHS performance (e.g., incident rates, LTIFR, near-miss reports).
    \item Compliance with legal requirements and certifications (e.g., COIDA registration).
    \item Resources and training programs for OHS.
\end{itemize}

\subsection{Contractual OHS Requirements}
OHS requirements shall be integrated into contracts, including:
\begin{itemize}
    \item Compliance with {{COMPANY_NAME}} OHS policies and legal standards.
    \item Submission of a Health and Safety File (Construction Regulations 2014).
    \item Penalties for OHS non-compliance, linked to the main project contract per JBCC Clause 12.2 (Penalties), NEC Clause 27.4 (Health and Safety Breaches), FIDIC Clause 4.8 (Safety Non-Compliance), and GCC Clause 4.9 (Contractor Responsibilities).
\end{itemize}

\subsection{Induction and Training}
Contractors shall receive site-specific OHS induction and any necessary training before commencing work, as per the OHS Training and Competency Standard.

\subsection{Monitoring and Supervision}
Contractor OHS performance shall be monitored through:
\begin{itemize}
    \item Regular site inspections and audits.
    \item Review of contractor-submitted incident reports and corrective actions.
    \item Compliance with their Health and Safety File and OHS plan.
\end{itemize}

\subsection{Incident Management}
Contractors shall report all incidents immediately, participate in investigations, and implement corrective actions. Incident data will be used to evaluate ongoing performance.

\subsection{Performance Evaluation}
Contractors shall be evaluated periodically using OHS performance metrics (e.g., incident frequency, audit scores, compliance rates). Poor performance may result in penalties or contract termination.

\subsection{Contract Close-Out}
At contract completion, a final OHS performance review shall be conducted, and contractors shall submit all required records (e.g., incident reports, training logs).

\section{Record Keeping}
Maintain records of contractor pre-qualification, contracts, OHS performance evaluations, incident reports, and training logs in a Contractor Register.

\section{Standard Review}
This standard will be reviewed at least every two years or as necessitated by changes in legislation, operations, or contractor performance trends.

\section{Signature}
\textbf{Approved By:}

{{CEO_NAME}}\\
\textbf{CEO, {{COMPANY_NAME}}}\\
\textbf{Date:} {{DATE}}

\section{Contact Information}
For inquiries, contact:
\begin{itemize}
    \item Phone: {{COMPANY_CONTACT}}
    \item Email: {{COMPANY_EMAIL}}
\end{itemize}

\section{Document Control}
\begin{table}[h]
    \centering
    \begin{tabular}{p{3cm}p{3cm}p{6cm}}
        \toprule
        \textbf{Version} & \textbf{Date} & \textbf{Changes} \\
        \midrule
        {{REVISION}} & {{DATE}} & Initial version \\
        \bottomrule
    \end{tabular}
    \caption{Revision History}
\end{table}

\section{Compiled By}
\begin{itemize}
    \item \textbf{Name}: {{COMPILED_BY_NAME}}
    \item \textbf{Contact}: {{COMPILED_BY_CONTACT}}
    \item \textbf{Email}: {{COMPILED_BY_EMAIL}}
    \item \textbf{Role}: {{COMPILED_BY_ROLE}}
\end{itemize}

\end{document}
