\documentclass[12pt]{article}
\usepackage[utf8]{inputenc}
\usepackage[T1]{fontenc}
\usepackage{geometry}
\geometry{a4paper, margin=1in}
\usepackage{enumitem}
\usepackage{booktabs}
\usepackage{hyperref}
\usepackage{titlesec}
\usepackage{noto}

\titleformat{\section}{\large\bfseries}{\thesection}{1em}{}
\titleformat{\subsection}{\normalsize\bfseries}{\thesubsection}{1em}{}
\titleformat{\subsubsection}{\normalsize\itshape}{\thesubsubsection}{1em}{}

\begin{document}

\begin{titlepage}
    \centering
    \vspace*{2cm}
    {\LARGE\bfseries Hazard Identification and Risk Assessment (HIRA) Procedure for {{COMPANY_NAME}}\par}
    \vspace{1cm}
    {\large\itshape Document Reference: HIRA-001\par}
    \vspace{0.5cm}
    {\normalsize Revision: {{REVISION}}\par}
    \vspace{0.5cm}
    {\normalsize Date: {{DATE}}\par}
    \vspace{2cm}
    {\normalsize Approved by: {{CEO_NAME}}\par}
\end{titlepage}

\section{Purpose}
This procedure outlines the process for identifying hazards and assessing risks at {{COMPANY_NAME}}, ensuring compliance with ISO 45001 Clause 6.1.2 and the Occupational Health and Safety Act 85 of 1993.

\section{Scope}
This procedure applies to all operations and activities at {{COMPANY_NAME}}'s premises located at {{COMPANY_ADDRESS}}.

\section{Procedure}
\subsection{Hazard Identification}
\begin{itemize}
    \item Identify potential hazards through workplace inspections, employee feedback, and incident reports.
    \item Categorize hazards (e.g., physical, chemical, biological).
\end{itemize}

\subsection{Risk Assessment}
\begin{itemize}
    \item Assess the likelihood and severity of each hazard.
    \item Use a risk matrix to determine risk levels.
\end{itemize}

\subsection{Controls}
\begin{itemize}
    \item Implement controls to eliminate or mitigate risks (e.g., engineering controls, PPE).
    \item Monitor and review controls regularly.
\end{itemize}

\section{Additional Details}
{{CUSTOM_CONTENT}}

\section{Contact Information}
For inquiries, contact:
\begin{itemize}
    \item Phone: {{COMPANY_CONTACT}}
    \item Email: {{COMPANY_EMAIL}}
\end{itemize}

\section{Document Control}
\begin{table}[h]
    \centering
    \begin{tabular}{p{3cm}p{3cm}p{6cm}}
        \toprule
        \textbf{Version} & \textbf{Date} & \textbf{Changes} \\
        \midrule
        {{REVISION}} & {{DATE}} & Initial version \\
        \bottomrule
    \end{tabular}
    \caption{Revision History}
\end{table}

\section{Compiled By}
\begin{itemize}
    \item \textbf{Name}: {{COMPILED_BY_NAME}}
    \item \textbf{Contact}: {{COMPILED_BY_CONTACT}}
    \item \textbf{Email}: {{COMPILED_BY_EMAIL}}
    \item \textbf{Role}: {{COMPILED_BY_ROLE}}
\end{itemize}

\end{document}
