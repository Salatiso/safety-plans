\documentclass[12pt]{article}
\usepackage[utf8]{inputenc}
\usepackage[T1]{fontenc}
\usepackage{geometry}
\geometry{a4paper, margin=1in}
\usepackage{enumitem}
\usepackage{booktabs}
\usepackage{hyperref}
\usepackage{titlesec}
\usepackage{noto}

\titleformat{\section}{\large\bfseries}{\thesection}{1em}{}
\titleformat{\subsection}{\normalsize\bfseries}{\thesubsection}{1em}{}
\titleformat{\subsubsection}{\normalsize\itshape}{\thesubsubsection}{1em}{}

\begin{document}

\begin{titlepage}
    \centering
    \vspace*{2cm}
    {\LARGE\bfseries Occupational Health and Safety (OHS) Environmental Management Standard for {{COMPANY_NAME}}\par}
    \vspace{1cm}
    {\large\itshape Standard Number: OHSEMS-001\par}
    \vspace{0.5cm}
    {\normalsize Version: {{REVISION}}\par}
    \vspace{0.5cm}
    {\normalsize Effective Date: {{DATE}}\par}
    \vspace{0.5cm}
    {\normalsize Review Date: {{REVIEW_DATE}}\par}
    \vspace{2cm}
    {\normalsize Approved by: {{CEO_NAME}}, CEO\par}
\end{titlepage}

\section{Introduction}
This Occupational Health and Safety (OHS) Environmental Management Standard provides the operational framework for implementing the {{COMPANY_NAME}} OHS Environmental Management Policy. It details processes for identifying, controlling, and mitigating environmental hazards that impact OHS, ensuring safe practices in areas like spill response, waste management, and air quality control.

\section{Purpose}
The purpose of this standard is to:
\begin{itemize}
    \item Define procedures for managing OHS-related environmental risks.
    \item Establish processes for spill response, waste management, and air quality monitoring.
    \item Ensure compliance with local legislation and SANS 45001.
    \item Protect worker health and safety from environmental hazards.
    \item Promote continual improvement in environmental OHS practices.
\end{itemize}

\section{Scope}
This standard applies to all {{COMPANY_NAME}} operations in South Africa where environmental factors may impact OHS.

\section{Definitions}
Refer to the {{COMPANY_NAME}} OHS Environmental Management Policy for key definitions. Additional definitions include:
\begin{itemize}
    \item \textbf{Air Quality Monitoring:} The process of measuring pollutants (e.g., dust, fumes) to ensure safe working conditions.
    \item \textbf{Hazardous Waste:} Waste that poses a risk to health (e.g., chemical residues, contaminated materials).
\end{itemize}

\section{Legal and Standard References}
This standard is guided by:
\begin{itemize}
    \item Occupational Health and Safety Act 85 of 1993 (Section 8).
    \item Environmental Regulations for Workplaces, 1987.
    \item Hazardous Chemical Substances Regulations (HCSR).
    \item SANS 45001: Occupational Health and Safety Management Systems (Clause 8.1.2).
\end{itemize}

\section{OHS Environmental Management Process}

\subsection{Environmental Risk Assessment}
\begin{itemize}
    \item Conduct environmental risk assessments as part of HIRAs, focusing on OHS impacts (e.g., chemical spills, air quality, waste exposure).
    \item Document risks and control measures in the risk register.
\end{itemize}

\subsection{Spill Response}
\begin{itemize}
    \item \textbf{Procedure:} Develop spill response plans, including containment, cleanup, and worker protection measures.
    \item \textbf{Training:} Train employees on spill response and use of spill kits.
    \item \textbf{Reporting:} Report spills impacting OHS as environmental incidents, per the Incident Management Standard.
\end{itemize}

\subsection{Waste Management}
\begin{itemize}
    \item \textbf{Segregation:} Separate hazardous and non-hazardous waste to minimize OHS risks.
    \item \textbf{Storage:** Store waste in designated, labeled areas to prevent exposure.
    \item \textbf{Disposal:** Ensure safe disposal by licensed contractors, retaining waste disposal certificates.
    \item \textbf{Checklist:** Use a Waste Management Checklist to verify compliance.
\end{itemize}

\subsection{Air Quality Management}
\begin{itemize}
    \item \textbf{Monitoring:** Measure air quality (e.g., dust, fumes) in high-risk areas, ensuring levels are below OELs.
    \item \textbf{Ventilation:** Maintain ventilation systems to control airborne hazards.
    \item \textbf{PPE:** Provide respiratory protection where air quality cannot be fully controlled.
\end{itemize}

\subsection{Hazardous Substance Handling}
\begin{itemize}
    \item \textbf{Storage:** Store hazardous substances per HCSR requirements (e.g., labeled containers, bunded areas).
    \item \textbf{Training:** Train employees on safe handling and emergency procedures.
    \item \textbf{MSDS:** Maintain accessible Material Safety Data Sheets for all substances.
\end{itemize}

\subsection{Training and Awareness}
Provide annual training on OHS-related environmental risks, including spill response, waste handling, and air quality safety.

\section{Record Keeping}
Maintain records of environmental risk assessments, spill reports, waste disposal certificates, and air quality monitoring results for at least 5 years.

\section{Standard Review}
This standard will be reviewed at least every two years or as necessitated by changes in legislation, operations, or environmental risks.

\section{Signature}
\textbf{Approved By:}

{{CEO_NAME}}\\
\textbf{CEO, {{COMPANY_NAME}}}\\
\textbf{Date:} {{DATE}}

\section{Contact Information}
For inquiries, contact:
\begin{itemize}
    \item Phone: {{COMPANY_CONTACT}}
    \item Email: {{COMPANY_EMAIL}}
\end{itemize}

\section{Document Control}
\begin{table}[h]
    \centering
    \begin{tabular}{p{3cm}p{3cm}p{6cm}}
        \toprule
        \textbf{Version} & \textbf{Date} & \textbf{Changes} \\
        \midrule
        {{REVISION}} & {{DATE}} & Initial version \\
        \bottomrule
    \end{tabular}
    \caption{Revision History}
\end{table}

\section{Compiled By}
\begin{itemize}
    \item \textbf{Name}: {{COMPILED_BY_NAME}}
    \item \textbf{Contact}: {{COMPILED_BY_CONTACT}}
    \item \textbf{Email}: {{COMPILED_BY_EMAIL}}
    \item \textbf{Role}: {{COMPILED_BY_ROLE}}
\end{itemize}

\end{document}
