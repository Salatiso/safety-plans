% Setting up the document class and essential packages
\documentclass[11pt]{article}
\usepackage[utf8]{inputenc}
\usepackage[a4paper, margin=2.5cm]{geometry}
\usepackage{fancyhdr}
\usepackage{draftwatermark}
\usepackage{tocbibind}
\usepackage{tabularx}
\usepackage{booktabs}
\usepackage{graphicx}
\usepackage{hyperref}
\usepackage{noto}
\usepackage{ifthen}

% Configuring branding variables for SafetyHelp
\newcommand{\brandingLevel}{light}
\newcommand{\userLogo}{}
\newcommand{\userCompany}{{{COMPANY_NAME}}}
\newcommand{\safetyHelpLogo}{/safety-plans/assets/images/logo.png}
\SetWatermarkText{SafetyHelp}
\SetWatermarkScale{2}
\SetWatermarkColor[gray]{0.9}

% Defining dynamic fields from wizard input
\newcommand{\docTitle}{Occupational Health and Safety Specification}
\newcommand{\refNumber}{OHSS-001}
\newcommand{\issueDate}{{{DATE}}}
\newcommand{\projectName}{{{PROJECT_NAME}}}
\newcommand{\clientName}{{{COMPANY_NAME}}}
\newcommand{\clientContact}{{{COMPANY_CONTACT}}, {{COMPANY_EMAIL}}}
\newcommand{\compilerName}{{{COMPILED_BY_NAME}}}
\newcommand{\compilerRole}{{{COMPILED_BY_ROLE}}}
\newcommand{\compilerSACPCMP}{}
\newcommand{\compilerEmail}{{{COMPILED_BY_EMAIL}}}
\newcommand{\compilerPhone}{{{COMPILED_BY_CONTACT}}}
\newcommand{\compilerCompany}{}
\newcommand{\revision}{{{REVISION}}}
\newcommand{\reviewDate}{{{REVIEW_DATE}}}
\newcommand{\site\documentclass[11pt]{article}
\usepackage[utf8]{inputenc}
\usepackage[a4paper, margin=2.5cm]{geometry}
\usepackage{fancyhdr}
\usepackage{draftwatermark}
\usepackage{tocbibind}
\usepackage{tabularx}
\usepackage{booktabs}
\usepackage{graphicx}
\usepackage{hyperref}
\usepackage{noto}
\usepackage{ifthen}

% Branding variables
\newcommand{\brandingLevel}{light}
\newcommand{\userLogo}{}
\newcommand{\userCompany}{{{COMPANY_NAME}}}
\newcommand{\safetyHelpLogo}{/safety-plans/assets/images/logo.png}
\SetWatermarkText{SafetyHelp}
\SetWatermarkScale{2}
\SetWatermarkColor[gray]{0.9}

% Dynamic fields
\newcommand{\docTitle}{Occupational Health and Safety Specification}
\newcommand{\refNumber}{OHSS-001}
\newcommand{\issueDate}{{{DATE}}}
\newcommand{\projectName}{{{PROJECT_NAME}}}
\newcommand{\clientName}{{{COMPANY_NAME}}}
\newcommand{\clientContact}{{{COMPANY_CONTACT}}, {{COMPANY_EMAIL}}}
\newcommand{\compilerName}{{{COMPILED_BY_NAME}}}
\newcommand{\compilerRole}{{{COMPILED_BY_ROLE}}}
\newcommand{\compilerSACPCMP}{}
\newcommand{\compilerEmail}{{{COMPILED_BY_EMAIL}}}
\newcommand{\compilerPhone}{{{COMPILED_BY_CONTACT}}}
\newcommand{\compilerCompany}{}
\newcommand{\revision}{{{REVISION}}}
\newcommand{\reviewDate}{{{REVIEW_DATE}}}

% Header and Footer
\pagestyle{fancy}
\fancyhf{}
\ifthenelse{\equal{\brandingLevel}{heavy}}{
  \fancyhead[L]{\small \refNumber}
  \fancyhead[C]{\ifthen\isempty{\userLogo}{\textbf{\userCompany}}{\includegraphics[height=1.5cm]{\userLogo}} \quad \includegraphics[height=1cm]{\safetyHelpLogo}}
  \fancyhead[R]{\small \compilerName}
  \SetWatermarkLightness{0.7}
}{
  \fancyhead[L]{\small \refNumber}
  \fancyhead[C]{\ifthen\isempty{\userLogo}{\textbf{\userCompany}}{\includegraphics[height=1.5cm]{\userLogo}}}
  \fancyhead[R]{\small \compilerName}
}
\ifthenelse{\equal{\brandingLevel}{none}}{
  \SetWatermarkText{}
}{}
\fancyfoot[L]{\small \textit{Powered by SafetyHelp | Safety Personalized, Compliance Simplified}}
\fancyfoot[C]{\small \textit{Visit safetyfirst.help | Email: salatiso@safetyfirst.help}}
\fancyfoot[R]{\small Page \thepage\ of \pageref{LastPage}}
\renewcommand{\headrulewidth}{0.4pt}
\renewcommand{\footrulewidth}{0pt}
\fancyfoot[C]{\parbox[b]{\textwidth}{\footnotesize \textit{This document is generated by SafetyHelp for OHS compliance. Verify applicability with local regulations.}}}

% Defining activity variables from wizard step 6
% Example: \def\activityScaffolding{} for selected activities
% These will be set dynamically based on wizard input
% Placeholder for demonstration; actual implementation will use wizard data
\def\activityScaffolding{}
\def\activityExcavation{}
% Add more as needed

\begin{document}

% Creating the cover page
\begin{titlepage}
  \centering
  \ifthenelse{\equal{\brandingLevel}{heavy} \OR \equal{\brandingLevel}{light}}{
    \includegraphics[height=1.5cm,keepaspectratio]{\safetyHelpLogo}\hfill
  }{}
  \vspace{2cm}
  \ifthen\isempty{\userLogo}{\textbf{\Large \userCompany}\vspace{1cm}}{\includegraphics[height=1.5cm]{\userLogo}\vspace{1cm}}
  {\Huge \textbf{\docTitle}}\vspace{1cm}
  {\Large Reference: \refNumber}\vspace{0.5cm}
  {\large Issue Date: \issueDate}\vspace{0.5cm}
  {\large Review Date: \reviewDate}\vspace{0.5cm}
  {\large Client: \clientName}\vspace{0.5cm}
  {\large \clientContact}\vspace{0.5cm}
  {\large Compiled By: \compilerName, \compilerRole}\vspace{0.5cm}
  {\large \compilerEmail, \compilerPhone}
  \vspace{2cm}
  \rule{\textwidth}{0.5pt}
\end{titlepage}

% Adding table of contents
\tableofcontents
\newpage

% Section 1: Introduction
\section{Introduction}
\subsection{Purpose}
This Health and Safety Specification (HSS) outlines the occupational health and safety requirements for the \projectName\ project, ensuring compliance with the \href{https://www.gov.za/documents/occupational-health-and-safety-act}{Occupational Health and Safety Act, 1993 (Act No. 85 of 1993)} and the \href{https://www.gov.za/documents/occupational-health-and-safety-act-construction-regulations-2014-7-feb-2014-0000}{Construction Regulations, 2014}. It aims to promote a safe and healthy working environment, minimize risks, and ensure all personnel are protected from workplace hazards.

\subsection{Scope}
The specification applies to all construction activities associated with the project, as defined by the project scope and activities selected in the SafetyHelp wizard. It serves as the basis for the Principal Contractor’s Health and Safety Plan and is an integral part of the contract.

\subsection{Implementation}
This document must be read in conjunction with the project scope of work and other contractual documents. No construction work may commence until the Principal Contractor’s Health and Safety Plan, based on this specification, is approved by the Client or their appointed Health and Safety Agent.

% Section 2: Project Description
\section{Project Description}
The \projectName\ project involves construction activities at \textbf{{{{SITE_ADDRESS}}}}, with an estimated duration of \textbf{{{{PROJECT_DURATION}}}} days and a budget of \textbf{R{{{{PROJECT_COST}}}}}. The project is managed by \textbf{\clientName}, with \textbf{\compilerName} (\compilerRole) overseeing health and safety compliance.

\subsection{Key Activities}
The following activities have been identified based on the project scope (selected in SafetyHelp wizard step 6):
\begin{itemize}
  \ifdefined\activityScaffolding
    \item Scaffolding
  \fi
  \ifdefined\activityExcavation
    \item Excavation
  \fi
  % Add more activities as per wizard input
\end{itemize}

% Section 3: Legal Requirements
\section{Legal Requirements}
This specification complies with the following legislation and regulations:
\begin{itemize}
  \item \href{https://www.gov.za/documents/occupational-health-and-safety-act}{Occupational Health and Safety Act, 1993 (Act No. 85 of 1993)}, as amended.
  \item \href{https://www.gov.za/documents/occupational-health-and-safety-act-construction-regulations-2014-7-feb-2014-0000}{Construction Regulations, 2014} (Government Notice No. R. 84 of 7 February 2014).
  \item \href{https://www.gov.za/documents/compensation-occupational-injuries-and-diseases-act}{Compensation for Occupational Injuries and Diseases Act, 1993 (Act No. 130 of 1993)}.
  \item SACPCMP Guidelines for Construction Health and Safety Professionals.
\end{itemize}
All parties must ensure compliance with these laws. The Principal Contractor is responsible for informing Contractors and Subcontractors of their legal obligations and implementing necessary measures.

% Section 4: Roles and Responsibilities
\section{Roles and Responsibilities}
\subsection{Client}
\textbf{\clientName} is responsible for:
\begin{itemize}
  \item Providing this Health and Safety Specification.
  \item Appointing a competent Health and Safety Agent (PrCHSA) to oversee compliance.
  \item Approving the Principal Contractor’s Health and Safety Plan.
\end{itemize}

\subsection{Principal Contractor}
The Principal Contractor is responsible for:
\begin{itemize}
  \item Developing and implementing a Health and Safety Plan based on this specification.
  \item Managing health and safety on site, including audits and inspections.
  \item Ensuring all Contractors and Subcontractors comply with the plan.
  \item Appointing a registered Construction Health and Safety Officer (CHSO) per SACPCMP requirements.
\end{itemize}

\subsection{Contractors and Subcontractors}
Contractors and Subcontractors must:
\begin{itemize}
  \item Comply with the Principal Contractor’s Health and Safety Plan.
  \item Conduct activity-specific risk assessments and implement control measures.
  \item Provide trained and competent personnel.
\end{itemize}

\subsection{Health and Safety Professionals}
\begin{itemize}
  \item \textbf{CHSO}: Conducts regular site inspections (minimum 4 visits/month), audits, and ensures compliance.
  \item \textbf{PrCHSA}: Acts on behalf of the Client, approves plans, and conducts bi-monthly audits.
\end{itemize}

% Section 5: Health and Safety Management
\section{Health and Safety Management}
\subsection{Health and Safety Plan}
The Principal Contractor must submit a Health and Safety Plan for approval, addressing all hazards identified in the risk assessment and including safe work procedures, as per Construction Regulation 7.

\subsection{Risk Assessment}
A comprehensive risk assessment must be conducted for all activities, identifying hazards and implementing control measures. Refer to SafetyHelp’s \href{https://safetyfirst.help/templates/ohs-system/risk-assessment-plan-template.tex}{Risk Assessment Plan Template}.

\subsection{Training and Competency}
All personnel must receive induction training and task-specific training. Competency records must be maintained in the Health and Safety File, per Construction Regulation 9.

\subsection{Incident Management}
Procedures for reporting, investigating, and managing incidents must be established, per Section 24 of the OHS Act. Use SafetyHelp’s \href{https://safetyfirst.help/templates/ohs-system/ohs-incident-report-investigation-template.tex}{Incident Report Template}.

\subsection{Audits and Inspections}
Regular audits and inspections must be conducted by the CHSO and PrCHSA, with reports maintained in the Health and Safety File, per Construction Regulation 5(1)(o).

% Section 6: Activity-Specific Requirements
\section{Activity-Specific Requirements}
\ifdefined\activityScaffolding
\subsection{Scaffolding}
\textbf{Description}: Erection, use, and dismantling of scaffolding for access and working at heights.

\textbf{Hazards and Risks}: Falls from height, scaffold collapse, falling objects.

\textbf{Control Measures}:
\begin{itemize}
  \item Scaffolding must be erected, altered, and dismantled by competent persons (Construction Regulation 16).
  \item Regular inspections by a competent person before use and weekly.
  \item Guardrails, toe boards, and access ladders required.
  \item Workers must use fall arrest systems.
\end{itemize}

\textbf{Legal Requirements}: Construction Regulation 16; SANS 10085-2004.
\fi

\ifdefined\activityExcavation
\subsection{Excavation}
\textbf{Description}: Digging and trenching for foundations, utilities, or other purposes.

\textbf{Hazards and Risks}: Cave-ins, falls, equipment accidents.

\textbf{Control Measures}:
\begin{itemize}
  \item Excavations >1m deep require shoring or battering (Construction Regulation 13).
  \item Daily inspections by a competent person.
  \item Barricades and signage to prevent unauthorized access.
\end{itemize}

\textbf{Legal Requirements}: Construction Regulation 13.
\fi
% Add more activity sections as per wizard input

% Section 7: General Safety Requirements
\section{General Safety Requirements}
\subsection{Personal Protective Equipment (PPE)}
All workers must wear appropriate PPE, including hard hats, safety boots, and high-visibility vests, as per Construction Regulation 23.

\subsection{Emergency Procedures}
A detailed emergency response plan must be in place, including evacuation procedures and contact information for emergency services, per Construction Regulation 29.

\subsection{First Aid}
Adequate first aid facilities and trained first aiders must be available, per Construction Regulation 3.

\subsection{Welfare Facilities}
Provide clean drinking water, sanitary facilities (1 toilet per 30 workers), and rest areas, per Construction Regulation 30.

\subsection{Housekeeping}
Maintain a clean and tidy site to prevent slips, trips, and falls, per Construction Regulation 27.

% Section 8: Monitoring and Compliance
\section{Monitoring and Compliance}
\subsection{Health and Safety File}
The Principal Contractor must maintain a Health and Safety File containing risk assessments, method statements, training records, and inspection reports, per Construction Regulation 7(1)(b).

\subsection{Reporting}
Regular health and safety performance reports must be submitted to the Client and authorities, per Section 24 of the OHS Act.

\subsection{Compliance Monitoring}
The Health and Safety Agent will conduct bi-monthly audits and inspections to ensure compliance, per Construction Regulation 5(1)(o).

% Section 9: Annexures
\section{Annexures}
\subsection{Non-Selected Activities}
For activities not selected in the project scope, refer to the following general guidelines:
\begin{itemize}
  \item \textbf{Scaffolding}: Comply with Construction Regulation 16 and SANS 10085-2004.
  \item \textbf{Excavation}: Follow Construction Regulation 13 for safe trenching practices.
  % Add more as needed
\end{itemize}

\subsection{Templates and Forms}
Use SafetyHelp templates for:
\begin{itemize}
  \item \href{https://safetyfirst.help/templates/ohs-system/risk-assessment-plan-template.tex}{Risk Assessment Plan}
  \item \href{https://safetyfirst.help/templates/ohs-system/section-16-1-appointment-template.tex}{Section 16(1) Appointment}
  % Add more as needed
\end{itemize}

\subsection{Baseline Risk Assessment}
A generic baseline risk assessment is included, covering common construction hazards.

\subsection{Legal Register}
A list of applicable legislation is provided for reference.

% Signature Page
\section{Signature Page}
\begin{itemize}
  \item \textbf{Compiler}: \compilerName \\
    Signature: \rule{5cm}{0.4pt} \quad Date: \rule{3cm}{0.4pt}
  \item \textbf{Approved By}: {{CEO_NAME}} \\
    Signature: \rule{5cm}{0.4pt} \quad Date: \rule{3cm}{0.4pt}
\end{itemize}

% Back Page
\ifthenelse{\equal{\brandingLevel}{heavy} \OR \equal{\brandingLevel}{light} \OR \equal{\brandingLevel}{none}}{
  \newpage
  \section*{About SafetyHelp}
  SafetyHelp is South Africa’s leading OHS platform, offering AI-powered tools for compliance, risk management, and safety education across industries like construction, mining, education, and healthcare.

  \textbf{Comprehensive Services}: \\
  \textit{OHS Tools}: \\
  \begin{itemize}
    \item \textit{Construction Safety}: Generate H\&S specifications, plans, risk assessments, legal appointments (OHSA, COIDA compliance).
    \item \textit{Risk Management Plans}: Step-by-step wizard with ISO 45001 integration.
    \item \textit{Document Library}: Policies, procedures, templates, permits, records (downloadable PDFs).
  \end{itemize}
  \textit{Training and Education}: \\
  \begin{itemize}
    \item \textit{Training Wizard}: Tailored programs by industry, audience, location (PDF guides, online videos Q3 2025).
    \item \textit{Training Provider Directory}: Connect with accredited providers.
  \end{itemize}
  \textit{Community Safety Tools}: \\
  \begin{itemize}
    \item \textit{Incident Reporting}: Report hazards (e.g., potholes, equipment issues) with photos, location, weather data, share via email/SMS/WhatsApp/X.
    \item \textit{Safety Checklists}: For spaza shops, schools, hospitals, taverns.
  \end{itemize}
  \textit{Multilingual Accessibility}: Supports 15 languages (English, Zulu, Xhosa, Afrikaans, Mandarin, etc.) for inclusivity. \\
  \textit{User Dashboard}: Profile management, saved resources (RMPs, incident reports), supplier directory with reviews. \\
  \textit{Upcoming Features}: \\
  \begin{itemize}
    \item Online Video Training (Q3 2025)
    \item Mobile App (iOS/Android, Q4 2025)
    \item AI Chatbot Enhancements (Q2 2026)
  \end{itemize}

  \textbf{Contact}: Visit \href{https://safetyfirst.help}{safetyfirst.help} or email \href{mailto:salatiso@safetyfirst.help}{salatiso@safetyfirst.help} to learn more. Become a member to customize documents and remove branding!

  \vspace{2cm}
  \hfill \includegraphics[height=2cm]{\safetyHelpLogo}
}{}

\label{LastPage}
\end{document}
