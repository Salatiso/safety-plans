\documentclass[12pt]{article}
\usepackage[utf8]{inputenc}
\usepackage[T1]{fontenc}
\usepackage{geometry}
\geometry{a4paper, margin=1in}
\usepackage{enumitem}
\usepackage{booktabs}
\usepackage{hyperref}
\usepackage{titlesec}
\usepackage{noto}

\titleformat{\section}{\large\bfseries}{\thesection}{1em}{}
\titleformat{\subsection}{\normalsize\bfseries}{\thesubsection}{1em}{}
\titleformat{\subsubsection}{\normalsize\itshape}{\thesubsubsection}{1em}{}

\begin{document}

\begin{titlepage}
    \centering
    \vspace*{2cm}
    {\LARGE\bfseries Occupational Health and Safety (OHS) Legal Compliance Management Standard for {{COMPANY_NAME}}\par}
    \vspace{1cm}
    {\large\itshape Standard Number: OHSLCS-001\par}
    \vspace{0.5cm}
    {\normalsize Version: {{REVISION}}\par}
    \vspace{0.5cm}
    {\normalsize Effective Date: {{DATE}}\par}
    \vspace{0.5cm}
    {\normalsize Review Date: {{REVIEW_DATE}}\par}
    \vspace{2cm}
    {\normalsize Approved by: {{CEO_NAME}}, CEO\par}
\end{titlepage}

\section{Introduction}
This Occupational Health and Safety (OHS) Legal Compliance Management Standard provides the operational framework and minimum requirements for implementing the {{COMPANY_NAME}} OHS Legal Compliance Policy. It outlines systematic processes for identifying, accessing, and managing OHS legal and other requirements, integrating them into the OHS Management System, and evaluating compliance.

\section{Purpose}
The purpose of this standard is to:
\begin{itemize}
    \item Define the methodology for identifying, accessing, and managing OHS legal requirements.
    \item Establish responsibilities for each stage of the compliance management process.
    \item Ensure a consistent approach to maintaining OHS legal compliance.
    \item Provide a framework for evaluating compliance and addressing non-compliances.
    \item Support continual improvement of OHS legal compliance performance.
\end{itemize}

\section{Scope}
This standard applies to all processes involved in the identification, management, and evaluation of OHS legal and other requirements across all {{COMPANY_NAME}} operations in South Africa.

\section{Definitions}
Refer to the {{COMPANY_NAME}} OHS Legal Compliance Policy for key definitions. Additional definitions include:
\begin{itemize}
    \item \textbf{Compliance Obligation:} A requirement that an organization must or chooses to comply with.
    \item \textbf{Evaluation of Compliance:} A systematic process to determine fulfillment of OHS legal requirements.
\end{itemize}

\section{Legal and Standard References}
This standard is guided by:
\begin{itemize}
    \item Occupational Health and Safety Act 85 of 1993.
    \item SANS 45001: Occupational Health and Safety Management Systems.
    \item ISO 37301: Compliance Management Systems.
    \item ISO 19011: Guidelines for Auditing Management Systems.
\end{itemize}

\section{OHS Legal Compliance Management Process}

\subsection{Identification of OHS Legal and Other Requirements}
The OHS Manager shall oversee the identification of OHS legal requirements using reliable sources (e.g., government websites, legal publishers). This process shall be reviewed annually or upon significant changes.

\subsection{Access to and Understanding of Requirements}
{{COMPANY_NAME}} shall maintain access to current legal requirements, interpret their applicability, and communicate them to affected personnel.

\subsection{Documenting OHS Legal and Other Requirements (Legal Register)}
A Legal Register shall be maintained, including the title, reference, summary, applicability, obligations, and responsible persons for each requirement.

\subsection{Integrating Requirements into the OHS Management System}
Legal requirements shall be integrated into OHS objectives, risk assessments, procedures, and training programs.

\subsection{Evaluation of Compliance}
Compliance shall be evaluated at planned intervals using audits, inspections, and document reviews. Results shall be documented.

\subsection{Management of Non-Compliance}
Non-compliances shall be reported, investigated, and corrected with assigned responsibilities and timelines.

\subsection{Training and Awareness}
Employees shall receive training on OHS legal requirements relevant to their roles.

\subsection{Management Review}
Compliance status shall be reviewed during OHS Management System reviews.

\section{Record Keeping}
Records shall include the Legal Register, compliance evaluations, non-compliance reports, corrective actions, and training records.

\section{Standard Review}
This standard will be reviewed at least every two years or as necessitated by changes in legislation or operations.

\section{Signature}
\textbf{Approved By:}

{{CEO_NAME}}\\
\textbf{CEO, {{COMPANY_NAME}}}\\
\textbf{Date:} {{DATE}}

\section{Contact Information}
For inquiries, contact:
\begin{itemize}
    \item Phone: {{COMPANY_CONTACT}}
    \item Email: {{COMPANY_EMAIL}}
\end{itemize}

\section{Document Control}
\begin{table}[h]
    \centering
    \begin{tabular}{p{3cm}p{3cm}p{6cm}}
        \toprule
        \textbf{Version} & \textbf{Date} & \textbf{Changes} \\
        \midrule
        {{REVISION}} & {{DATE}} & Initial version \\
        \bottomrule
    \end{tabular}
    \caption{Revision History}
\end{table}

\section{Compiled By}
\begin{itemize}
    \item \textbf{Name}: {{COMPILED_BY_NAME}}
    \item \textbf{Contact}: {{COMPILED_BY_CONTACT}}
    \item \textbf{Email}: {{COMPILED_BY_EMAIL}}
    \item \textbf{Role}: {{COMPILED_BY_ROLE}}
\end{itemize}

\end{document}
